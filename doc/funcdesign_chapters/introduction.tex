\section{Purpose and scope of this document}
This document concerns the \textit{Functional design} of the \probLib \textbf{version \probLibVersion{}}. For the requirements for the \probLib, a distinction is made between functional requirements (\textit{specific behaviors: what should the software actually be able to do?}) and non-functional requirements (\textit{specifying criteria that can be used to judge the operation of a system, rather than specific behaviors}). Since the \probLib was part of \hr, the requirements for the \probLib overlap with the requirements imposed on \hr.

\section{Scientific description of the \probLib}
The scientific description of the \probLib is contained in the following appendices:
\begin{itemize}
\item Probabilistic computation techniques for system reliability -- \Aref{chap:systemreliability},
\item Statistical distribution functions -- \Aref{Section_3.3},
\item Correlation models -- \Aref{Section_3.4},
\item Conversion functions for the standard normal distribution -- \Aref{appConversions}.
\end{itemize}
\Aref{chap:systemreliability} - \Aref{Section_3.4} come from \cite{TechRef}. Whereas \Aref{appConversions} is based on \cite{Vrieling_2017}. In particular, \cite{TechRef} presents a scientific description of \hr. Since the \probLib was part of \hr, the scientific description of the \probLib overlaps with the scientific description of \hr. 

\section{Relation to other documents}
The following documents describe the \probLib:
\begin{itemize}
\item \textit{Functional design}, with the description of the functional and non-functional requirements for the \probLib.
\item \textit{Technical design}, with the technical description of the main components of the \probLib.
\item \textit{Test plan}, with the test strategy for the \probLib.
\item \textit{Test report}, with the test results corresponding to the released version of the \probLib.
\end{itemize}
The (non-)functional requirements presented in this document originate from the scientific description of the \probLib (see \cite{TechRef} and \cite{Vrieling_2017}) as well as from the functional design of \hr (see \cite{FuncDesign}).

\section{Outline}
The functional and non-functional requirements for the \probLib are presented in \Sref{chap_requirements}. The scientific description of the \probLib is presented in \Aref{chap:systemreliability} - \Aref{appConversions}.