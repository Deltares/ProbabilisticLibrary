\chapter{Output uncertainty}
\label{sec:OutputUncertainty}

\section{Numerical integration}

Numerical integration varies all input parameters step by step and for each step combination, a realization is generated and calculated. Realization results are collected in bins, which form a histogram distribution.

\section{Crude Monte Carlo}

Crude Monte Carlo is the 'standard' Monte Carlo simulation. Random realizations will be generated and realization results are categorized into bins. When a bin gets more realizations, it has a higher probability.

The Crude Monte Carlo simulation results in a histogram distribution (see  \autoref{sec:HistogramDistribution}) or, if all model results are equal, a determisitic distribution (see \autoref{sec:DeterministicDistribution}). The distribution is fitted with all model results. 
 
\section{Importance sampling}

Importance sampling modifies randomly selected realizations in such a way that more realizations are selected in an area selected by the user. Therefore each realization is translated to  another realization. This is done in the same way as Importance Sampling for reliability., see \autoref{sec:ImportanceSampling}.

Importance sampling is useful when the user is interested in the tail of the distribution.

\section{FORM}

FORM (First Order Reliability Method) is like FORM in reliability analysis (see \autoref{sec:FORM}). Starting from the origin in u-apce, steps of fixed length are taken along the steepest gradient. For each step a realization is performed and its result is added to a CDF curve (see \autoref{sec:CDFCurveDistribution}).

Each subsequent step in the resulting CDF curve is calculated as follows:

\begin{align}
\label{eq:FORMCDF}
\vec{u}_{i+1} = \vec{u}_{i} + L \cdot \nabla z \left(\vec{u_{i}}\right)
\end{align}

The reliability of the added point is

\begin{align}
\label{eq:FORMReliability}
\beta_{i+1} = \mid\mid\vec{u}_{i+1}\mid\mid
\end{align}

where
\begin{longtable*}{p{20mm}p{\textwidth-24pt-20mm}}  
	$\vec{u}$ & is a point in the parameter space defined in u-values; \\  
	$z\left(\vec{u}\right)$ & is the z value calculated at $\vec{u}$; \\  
	$L$ & is the step size; \\  
	$\nabla z\left(\vec{u}\right)$ & is the steepest gradient of $z$ at $\vec u$. \\
\end{longtable*}

\section{FOSM}

FOSM (First Order Second Moment) is a much faster technique than Crude Monte Carlo. It takes the gradient in the origin and then it predicts the further shape of the result distributions. It assumes the results have a normal distribution.

Although very fast, its assumptions than results have a normal distribution is often not true. But for a quick feeling of the output uncertainty it can be very useful.

