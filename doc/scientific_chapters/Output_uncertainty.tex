\chapter{Output uncertainty}
\label{chp:OutputUncertainty}

Output uncertainty is an extension of sensitivity analysis. In this approach, all input parameters are varied according to their defined uncertainties, leading to uncertainty in the output parameters.

This method is useful when the user is interested in the potential future values of a physical property. For example, due to a load, subsidence of the soil surface may occur. It would be useful to know how much subsidence could take place over the next ten years. Another example is the movement of a gully due to a side stream. In this case, it is important to predict the gully's location in the coming year.

This chapter presents the methods available in the \probLib for output uncertainty analysis.

\section{Numerical integration}

Numerical integration varies all input parameters step by step, generating and calculating a realization for each step combination. The resulting realizations are collected into bins, forming a histogram distribution.

\section{Crude Monte Carlo}

Crude Monte Carlo is the 'standard' Monte Carlo simulation method. Random realizations are generated, and their results are categorized into bins. The more realizations a bin contains, the higher its estimated probability.

The Crude Monte Carlo simulation produces a histogram distribution (see \autoref{sec:HistogramDistribution}) or, if all model results are identical, a deterministic distribution (see \autoref{sec:DeterministicDistribution}). The distribution is fitted using all model results.

\section{Importance sampling}

Importance sampling modifies randomly selected realizations so that more realizations are concentrated in a user-specified region. Each realization is transformed into another realization using the same approach as Importance Sampling for reliability (see \autoref{sec:ImportanceSampling}).

This method is particularly useful when the user is interested in the tail of the distribution.

\section{Directional sampling}

Directional sampling is an accelerated sampling method that identifies the failure domain by sampling along randomly chosen directions. The method operates in a transformed standard space with standard Gaussian variables (mean = $0$, standard deviation = $1$).

After transferring the model into the standard space, the value of model $Z_0$ is calculated for a sample at the center of space $u_0$. 

$n$ points are randomly drawn in the variable space following a uniform distribution. The lines connecting these points to the origin serve as the random directions. In the next step, the required reliability index $\beta_{q}$ corresponding to the given quantile value or probability of exceedance, is evaluated and used to determine the distance $d_{i,0} = |\beta_{q}|$ from the origin. The distance is then used to obtain $n$ sample points in the sample space.

Now, the value of the model $Z_{i,1}$ at each sample point ($d_{i,0}$ distance from the origin on each direction) is evaluated. Subsequently, the related $\beta$ value for each samples (i.e. directions) is calculated. If $\beta_{q}$ is greater than $\beta_0$, every direction with $Z_{i,1} < Z0$ is disregarded in the future probability calculations (i.e., they proceed in an undesired direction), otherwise, they are considered valid. Note that the number of samples $n$ is independent of their validity status. To calculate the probability of failure, equation \autoref{eq:DirectionalSamplingWeight} will be used. At this stage, the assumption of $\beta_{dir} = d_{i,0}$ for all directions is applied. See \ref{fig:DS-first} as the first step for generating the directions and samples, shown for the model $Z=X1_X2$, with $X1$ and $X2=N(0,1)$.
\begin{figure}[H]
	\centering
	\includegraphics[width=0.8\textwidth]{\pathProbLibPictures DS-first.png}
	\caption{Generation of first samples in random directions}
	\label{fig:DS-first}
\end{figure}

The main goal is to find the same value of $Z$ in all directions, with a given tolerance, as the model response for a given exceedance probability. If the $Z$ are not the same for all directions, the following steps must be followed to achieve consistency.

\textbf{[1]} - For each direction $i$, determine a new distance $d_{i,2}$ from the origin such that it satisfies the following condition:

\begin{itemize}
	\item the predicted model value $Z_{pred}$ is the same for all directions (prediction based on interpolation between $Z_0$ and $max(Z_{i,1})$) ;
	\item the combined reliability, based on equation \autoref{eq:DirectionalSampling}, remains equal to $\beta_q$. Note that even when a direction is disregarded in previous steps, still the number of directions $n$, is not changed. 
\end{itemize}

\textbf{[2]} - $d_{i,2}$ is calculated based on the interpolation between $Z_{pred}$ and $Z_{i,1}$ values. Now the model response $Z_{i,2}$ at distance $d_{i,2}$ is evaluated. 

The final steps will be repeated to reach $Z_{i}$, either up to a maximum of $𝑁$ iterations or until the convergence criteria are met. The final $Z_{i}$ then used as the output of the model at the given quantile value or probability of exceedance. The graph \ref{fig:DS-last} shows the output for 90\% quantile, where $Z=1,3826$ is evaluated. The reader may notice how even a limited number of valid samples shape the borders of the failure area for the given exceedence quantile, forming a linear line for the simple model of $Z=X1 + X2$. 

\begin{figure}[H]
	\centering
	\includegraphics[width=0.8\textwidth]{\pathProbLibPictures DS-last.png}
	\caption{The second/final step of iterations for the valid directions}
	\label{fig:DS-last}
\end{figure}

For the error criteria, the distribution of differences between the distances to the center of the $U$-space in all valid directions from the last and current iterations is considered. These differences form a distribution, and its 95\% quantile is used as an indicator of convergence. This quantile is then compared with a user-defined threshold.

%It worthy to mention that the directions are disregarded in the analysis, if:
%
%\begin{itemize}
%	\item calculated distance for them is much higher than $/beta_q$ or
%	\item calculated distance for them is negative
%\end{itemize}

\section{FORM}

FORM (First Order Reliability Method) is like FORM in reliability analysis (see \autoref{sec:FORM}).
Starting from the origin in $u$-space, steps of fixed length are taken along the steepest gradient.
For each step a realization is performed and its result is added to a CDF curve
(see \autoref{sec:CDFCurveDistribution}).

Each subsequent step in the resulting CDF curve is calculated as follows:

\begin{align}
\label{eq:FORMCDF}
\vec{u}_{i+1} = \vec{u}_{i} + L \cdot \nabla z \left(\vec{u_{i}}\right)
\end{align}

The reliability of the added point is:

\begin{align}
\label{eq:FORMReliability}
\beta_{i+1} = \mid\mid\vec{u}_{i+1}\mid\mid
\end{align}

where:
\begin{longtable*}{p{20mm}p{\textwidth-24pt-20mm}}  
	$\vec{u}$ & is a point in the parameter space defined in u-values \\  
	$z\left(\vec{u}\right)$ & is the z value calculated at $\vec{u}$ \\  
	$L$ & is the step size \\  
	$\nabla z\left(\vec{u}\right)$ & is the steepest gradient of $z$ at $\vec u$ \\
\end{longtable*}

\section{FOSM}

FOSM (First Order Second Moment) is a much faster technique than Crude Monte Carlo. It calculates the gradient at the origin and then predicts the shape of the resulting distributions. It assumes that the results follow a normal distribution.

Although very fast, its assumption that results have a normal distribution is often not true. However, for quickly estimating the output uncertainty, it can be very useful.
