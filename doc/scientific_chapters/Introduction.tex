\chapter{Introduction}

The \probLib provides functions for performing reliability and uncertainty analyses on any model, ranging from Python scripts to dedicated applications in the geotechnical or hydrodynamical fields, as well as combinations of models from various disciplines.

Reliability analysis determines the reliability -- or probability of failure -- of a physical structure. This analysis provides insight into the likelihood of an unwanted event occurring within a given period. For example, the probability of dike failure can be calculated, which is essential for dike assessments. Based on this probability, decisions can be made regarding whether reinforcement is necessary. Accurate failure probability calculations are crucial, as they help avoid unnecessary strengthening, thereby saving costs.

To improve the accuracy of failure probability calculations, survived events can be incorporated. Even with extensive efforts to measure input values, it is impossible to fully characterize the subsoil along long dike stretches. As a result, a model may predict dike failure, while real-world observations show that the dike has survived high water events. The \probLib integrates these observed events to refine and update failure probability estimates.

Another application of the  \probLib is risk-based asset management. In this approach, a maintenance schedule is developed to minimize total costs over an extended period of time. These costs include both strengthening actions and risk-related expenses, where risk is defined as the probability of failure multiplied by the associated damage.

This document describes the scientific background of the \probLib \probLibVersion.

\section{Overview of this document}

\Autoref{chp:terminology} defines the terminology used throughout this document. \Autoref{chp:ReliabilityAnalysis} and \autoref{chp:CombiningFailureProbabilities} describe the reliability techniques, while \autoref{chp:upscaling} explains the upscaling methods. The available distribution functions are provided in \autoref{chp:distributions}. The correlation models are described in \autoref{chp:Correlations}. \Autoref{chp:Sensitivity} and \autoref{chp:OutputUncertainty} present the techniques for sensitivity and uncertainty analyses, respectively. 