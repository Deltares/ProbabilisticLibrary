\chapter{Correlations}
\label{chp:Correlations}

This chapter describes how correlations are handled in the \probLib. A theoretical background on the available correlation models can be found in \Aref{chp:CorrelationModels}.

\section{Correlation factor}
\label{sec:CorrelationFactor}

The correlation factor $\rho_{\text{pq}}$ between two random variables, $p$ and $q$, indicates the degree to which they are related. The correlation can range from $-1$ (fully anti-correlated) to $0$ (uncorrelated) to $1$ (fully correlated). Values between these extremes represent partial correlation.

The Pearson correlation factor, which is used in \probLib, is calculated as follows:

\begin{align}
\label{eq:CorrelationCoefficient}
\rho_{\text{p,q}} = \frac{\sum_i u_{\text{p,i}} u_{\text{q,i}}}{\sum_i u_{\text{p,i}} \sum_i u_{\text{q,i}}}
\end{align}

where:
\begin{longtable*}{p{20mm}p{\textwidth-24pt-20mm}}  
	$i$ & is the index number of the observed value \\  
	$p$, $q$ & indicate random variables \\
	$u_{\text{p,i}}$ & is the $u$-value corresponding with observed value $x_i$ (see \autoref{eq:XUConversion}, note that the distribution of variable $p$ is derived already) \\  
\end{longtable*}

\subsection{Distance based correlation factor}
\label{sec:DistanceBasedCorrelationFactor}

If two components with the same variables have a location, the correlation between these variables can be determined based on their residual correlation and their distance, rather than using \autoref{eq:CorrelationCoefficient}. The correlation factor is then calculated as follows:

\begin{align}
\label{eq:DistanceCorrelationCoefficient}
\rho_{\text{t,p,q}} = \rho_{\text{t, rest}} + \left(1 -\rho_{\text{t, rest}}\right) \cdot \exp\left(  {-\frac{D_{\text{p, q}}^2}{d_{\text{t}}^2}}\right)
\end{align}

where:
\begin{longtable*}{p{20mm}p{\textwidth-24pt-20mm}}  
	$t$ & indicates a variable type\\  
	$p$, $q$ & indicate components \\
	$\rho_{\text{t, rest}}$ & is the user provided rest correlation for a certain variable type $t$ \\  
	$D_{\text{p, q}}$ & is the distance between components $p$ and $q$\\  
	$d_{\text{t}}$ & is the correlation length of variable type $t$ \\  
\end{longtable*}


\section{Processing of correlation factors}

Several probabilistic techniques are based on uncorrelated $u$-values. To account for the effect of correlations, model evaluations should not use the $x$-values directly converted from $u$, but rather a transformed counterpart of $u$ that incorporates the correlations. In other words, before a model evaluation is carried out, $u_\text{uncorrelated}$ is converted to $u_\text{correlated}$. This is done as follows:

\begin{align}
u_\text{correlated} = L\left(u_\text{uncorrelated}\right)
\end{align}

where $L$ denotes the lower triangular matrix obtained from the Cholesky decomposition of the correlation matrix $\left[\rho\right] = LL^\top$.