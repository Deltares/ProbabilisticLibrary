\chapter{Upscaling for systems with identical components: numerical integration with constant correlation}\label{chp:upscaling}

Upscaling refers to combining failure probabilities over ''identical components''.
Upscaling is distinguished from the more generic Hohenbichler combination techniques because the components being identical allows for a more convenient and efficient combination procedure.
Identical in this case refers to the fact that the components have the same failure probability (i.e.\ the same reliability index $\beta $) and they are mutually correlated with the same correlation coefficient $\rho$: 
\begin{align}
\begin{array}{l} {\beta \left(Z_{i} \right)\quad    =\beta \quad ;i=1...n_{e} } \\ {\rho \left(Z_{i} ,Z_{j} \right)=\rho \quad ;i\ne j} \end{array}\label{ZEqnNum897176} 
\end{align}
where $n_e$ is the number of components, $\rho$ is the correlation coefficient ($\rho \geqslant 0$) and $Z_i$ is the $Z$-function of component $i$.
Note that in general, the components also have in common the underlying set of random variables and the associated $\alpha$-values, but that is not a necessary condition for applying the method as described in this section.

Examples of when upscaling may be applied are the combining of failure probabilities at one time scale to a larger time scale  and upscaling failure probabilities from a cross section of a defense segment to the longitudinal extent of the segment.
Such applications are described in \Aref{Section_2.5.5}.
The value of $\rho$ first needs to be determined from system knowledge.
For now, $\rho$ is assumed to be known.

\section{Computational procedure}\label{Section_2.4.3.1}

The failure probability of this system can be computed by solving the following integral numerically:
\begin{align}
P(F)=\int _{v}^{}\left[1-\left\{1-\Phi \left(-\beta^{*}\right)\right\}^{n} \right]\phi (v)dv  \label{0.145)} 
\end{align}

In which $\phi$ is the standard normal density function and $\beta^{*}$ is equal to:
\begin{align}
\beta ^{*} =\frac{\beta -v\sqrt{\rho } }{\sqrt{1-\rho } } \label{0.144)} 
\end{align}

\subsection{Detailed explanation}\label{Section_2.4.3.2}

The upscaling method makes use of linearized approximations of the $Z$-functions, as described in \Aref{Section_2.2.5}.
The estimated probability of failure of the system is therefore an approximation of the true probability of failure.
Errors made in the approximation will depend on the system under condideration.
In \Aref{Section_2.2.5} it was shown that linearised $Z$-functions can be described as follows:
\begin{align}
Z_{i} =\beta -\alpha _{1} U_{i1} -...-\alpha _{n} U_{in} \quad ;i=1...n_{e} \label{ZEqnNum573383} 
\end{align}

Furthermore, it was shown that the sum of the product of $\alpha$-values and standard normally distributed $U$-values can be replaced by a single standard normally distributed $U$-variable:
\begin{align}
Z_{i} =\beta -U_{i} \quad ;i=1...n_{e} \label{ZEqnNum310338} 
\end{align}
where $U_i$ is a standard normally distributed variable and $\beta$ is the reliability index of each individual $Z$-function.
The value of $\beta$ is considered to be known, i.e.\ it is determined by the probabilistic computation techniques as described in \Aref{chp:ReliabilityAnalysis}.
This means $\beta$ is a constant in equation \eqref{ZEqnNum310338} and the mutual correlation of the $Z$-functions is therefore entirely determined by the mutual correlation of the $U$-variables:
\begin{align}
\rho \left(Z_{i} ,Z_{j} \right)=\rho \left(U_{i} ,U_{j} \right)=\rho \quad ;i\ne j\label{ZEqnNum614884} 
\end{align}

To describe a system that satisfies the relation of equation \eqref{ZEqnNum614884}, variable $U_i$ is written as a function of two independent standard normal random variables $U_i^{*}$ and $V$:
\begin{align}
U_{i} =U_{i}^{*} \sqrt{1-\rho } -V\sqrt{\rho } \label{ZEqnNum725764} 
\end{align}

The variables $U_i^{*}$, $i=1...n$ are taken to be mutually independent:
\begin{align}
\rho \left(U_{_{i} }^{*} ,U_{_{j} }^{*} \right)=0\quad ;i\ne j\label{ZEqnNum257919} 
\end{align}

%\Note{there is a difference between equation \eqref{ZEqnNum725764} and equation \eqref{eq:2-80}, where $\rho$ is used instead of $\sqrt{\rho}$, even though in both cases the correlation between $U_1$ and $U_2$ is equal to $\rho$.
%The reason for this difference is the fact that in equation \eqref{eq:2-80}, variable $U_2$ is written as a function of $U_1$, whereas in this case, $U_1$ and $U_2$ are both written as a function of a separate variable $V$.}

\Note{in the equations above, $\rho$ $>$ $0$.
For $\rho$ = $0$, the Hohenbichler method in combination with the outcrossing approach should be used.
The outcrossing method is discussed further on in this chapter.}

To verify the applicability of equation \eqref{ZEqnNum725764} it needs to be shown that [1] $U_i$ is standard normally distributed and [2] that the relation of equation \eqref{ZEqnNum614884} holds.
To prove [1], we apply the following general rule (see, e.g. \cite{GrimmettStirzaker1983}): If $X$ and $Y$ are independent normally distributed random variables, then $aX+bY$ is also normally distributed with a mean, $\mu$, and standard deviation, $\sigma$, equal to:
\begin{align}
\begin{array}{l} {\mu =a\mu _{X} +b\mu _{Y} } \\ {\sigma =\sqrt{a^{2} \sigma _{X}^{2} +b^{2} \sigma _{Y}^{2} } } \end{array} 
\end{align}

Application of this rule on equation \eqref{ZEqnNum725764}, where $U_i^{*}$ and $V$ are both normally distributed with mean $0$ and standard deviation $1$, gives:
\begin{align}
\begin{array}{l} {\mu =\sqrt{1-\rho } \cdot 0-\sqrt{\rho } \cdot 0=0} \\ {\sigma =\sqrt{\left(1-\rho \right)\cdot 1+\rho \cdot 1} =1} \end{array} 
\end{align}

Which proves that $U_i$ is standard normally distributed.
To prove [2], i.e.\ that the relation of equation \eqref{ZEqnNum614884} holds, equations \eqref{ZEqnNum310338} and \eqref{ZEqnNum725764} are combined: 
\begin{align}
Z_{i} =\beta -U_{i}^{*} \sqrt{1-\rho } -V\sqrt{\rho } \quad ;i=1...n_{e} \label{ZEqnNum883030} 
\end{align}

The variable $V$ in equation \eqref{ZEqnNum883030} is part of each $Z$-function.
This creates the desired mutual correlation between the functions $Z_i$, $i=1...n_e$.
To prove this, it needs to be shown that variables $U_i$ and $U_j$, $i\neq j$, have a mutual correlation equal to $\rho$.
The correlation between $U_i$ and $U_j$ is derived as follows:
\begin{align}
\rho \left(U_{i} ,U_{j} \right)=\frac{cov\left(U_{i} ,U_{j} \right)}{\left[\sigma \left(U_{i} \right)\sigma \left(U_{j} \right)\right]} =\frac{cov\left(U_{i} ,U_{j} \right)}{\left[1\cdot 1\right]} =cov\left(U_{i} ,U_{j} \right)\label{0.134)} 
\end{align}

The covariance of $U_i$ and $U_j$ is equal to:
\begin{align}
cov\left(U_{i} ,U_{j} \right) \begin{array}{l} {=E\left[U_{i} U_{j} -\mu \left(U_{i} \right)\mu \left(U_{j} \right)\right]=E\left[U_{i} U_{j} \right]} \\ {=E\left[\left(U_{i}^{*} \sqrt{1-\rho } -V\sqrt{\rho } \right)\left(U_{j}^{*} \sqrt{1-\rho } -V\sqrt{\rho } \right)\right]} \\ {=E\left[U_{i}^{*} U_{j}^{*} \left(1-\rho \right)-U_{i}^{*} V\sqrt{\rho \left(1-\rho \right)} -U_{j}^{*} V\sqrt{\rho \left(1-\rho \right)} +\rho V^{2} \right]} \\ {=E\left[0-0-0+\rho V^{2} \right]=\rho E\left[V^{2} \right]=\rho Var\left(V\right)=\rho } \end{array}\label{0.135)} 
\end{align}

This proves that equation \eqref{ZEqnNum883030} describes a system of $n_{e}$ components that are mutually correlated with a correlation coefficient $\rho$.
The theorem of total probability is used to derive the probability of failure of this system: 
\begin{align}
P(F)=P\left(Z_{1} <0  \cup   \dots  \cup Z_{n} <0\right)=\int _{v}^{}P\left(Z_{1} <0  \cup   \dots  \cup Z_{n} <0|v\right)\phi \left(v\right)dv \label{ZEqnNum846568} 
\end{align}
where $\phi(v)$ is the probability density function of the standard normal distribution.
The probability that at least $1$ component fails is equal to $1$ minus the probability that none of the components fails.
Equation \eqref{ZEqnNum846568} can therefore be rewritten as:
\begin{align}
\begin{array}{l} {P(F)=\int _{v}^{}\left[1-P\left(Z_{1} \ge 0  \cap     \dots  \cap Z_{n} \ge 0|v\right)\right]\phi \left(v\right)dv } \\ {\quad \quad   =\int _{v}^{}\left[1-P\left\{\left(Z_{1} \ge 0|v\right)\cap  \dots  \cap  \left(Z_{n} \ge 0|v\right)\right\}\right]\phi \left(v\right)dv } \end{array}\label{ZEqnNum753473} 
\end{align}

For a given value of $V$, the individual failure probabilities of the $Z$-functions are mutually independent:
\begin{align}
P\left[\left(Z_{i} <0|v\right)\cap \left(Z_{j} <0|v\right)\right]=P\left(Z_{i} <0|v\right)P\left(Z_{j} <0|v\right)\quad ;i\ne j\label{ZEqnNum938061} 
\end{align}

This can be easily verified from equation \eqref{ZEqnNum725764}.
If the value of $V$ is given, equation \eqref{ZEqnNum725764} only contains one random variable: $U_i^{*}$.
Since the $U_i^{*}$-values are mutually independent (see equation \eqref{ZEqnNum257919}), the $Z$-functions of equation \eqref{ZEqnNum725764} are mutually independent as well, which leads to the equality in equation \eqref{ZEqnNum938061}.
Substitution of equation \eqref{ZEqnNum938061} in equation \eqref{ZEqnNum753473} gives:
\begin{align}
P\left(F\right)=\int _{v}\left[1-\prod _{i=1}^{n_{e} }P\left(Z_{i} \ge 0  |v\right) \right]  \phi \left(v\right) dv\label{ZEqnNum169397} 
\end{align}

Because all components are identical, the following is true:
\begin{align}
P\left(Z_{1} \ge 0|v\right)=P\left(Z_{2} \ge 0|v\right)=...=P\left(Z_{n} \ge 0|v\right)\begin{array}{c} {\text{def}} \\ {=} \\ {} \end{array}P\left(Z\ge 0|v\right)\label{0.140)} 
\end{align}

This changes equation \eqref{ZEqnNum169397} into:
\begin{align}
P\left(F\right)=\int _{v}\left[1-P\left(Z\ge 0  |v\right)^{n_{e} } \right]  \phi \left(v\right)dv\label{ZEqnNum717936} 
\end{align}

The conditional probability of the $Z$-function in the integral is equal to: 
\begin{align}
P\left(Z\ge 0|v\right)=P\left(\beta -U^{*} \sqrt{1-\rho } -v\sqrt{\rho } \ge 0\right)=P\left(U^{*} \le \frac{\beta -v\sqrt{\rho } }{\sqrt{1-\rho } } \right)\label{ZEqnNum225981} 
\end{align}

Since $\nu$ is a given constant, $U^{*}$ is the only random variable in equation \eqref{ZEqnNum225981}, and since $U^{*}$ is standard normally distributed, the conditional probability is defined as:
\begin{align}
P\left(Z\ge 0|v\right)=\Phi \left(\frac{\beta -v\sqrt{\rho } }{\sqrt{1-\rho } } \right)\begin{array}{c} {\text{def}} \\ {=} \\ {} \end{array}\Phi \left(\beta ^{*} \right)\label{0.143)} 
\end{align}
where $\Phi$ is the standard normal distribution function and $\beta$ is equal to:
\begin{align}
\beta ^{*} =\frac{\beta -v\sqrt{\rho } }{\sqrt{1-\rho } } \label{0.144)} 
\end{align}

Equation \eqref{ZEqnNum717936} then changes into: 
\begin{align}
P(F)=\int _{v}^{}\left[1-\left\{1-\Phi \left(-\beta^{*}\right)\right\}^{n} \right]\phi (v)dv \label{ZEqnNum147042} 
\end{align}

The right hand side of equation \eqref{ZEqnNum147042} can be computed by numerical integration over the standard normally distributed variable $V$.
Since $V$ is the only variable, the grid size of $V$ can be chosen small without requiring significant computation time.
The error from the numerical integration of equation \eqref{ZEqnNum147042} can therefore be made as small as desired.
This means the only potentially significant error that is introduced in this method is related to the linearisation of the $Z$-function, which was necessary to derive equation \eqref{ZEqnNum147042}.

\subsection{Equivalent $\alpha$-values}\label{Section_2.4.3.3}

As with the Hohenbichler method, equivalent $\alpha$-values can be computed for the component that represents the combination of $n_{e}$ identical components.
This is necessary in case the resulting component is used in subsequent combining procedures where $\alpha$-values are required.
A similar approach with perturbed $u$-values is used as in the Hohenbichler method.
However, because in this special case the components are identical, this allows for some convenient simplifications that require less computation time.

\paragraph*{Computational procedure}

The method for deriving equivalent $\alpha$-values for the systems with identical components is as follows:

\textbf{[1]} Apply the upscaling method of \Aref{Section_2.4.3.1} to $n_{e}$ components with reliability index $\beta$ and mutual correlation $\rho$ to derive the reliability index $\beta^e$ of the combined (upscaled) component.

\textbf{[2]} Apply the upscaling method of \Aref{Section_2.4.3.1} on $n_{e}$ components with reliability index   $\beta$-$\varepsilon$$\surd$$\rho$ and mutual correlation $\rho$ to derive the reliability index $\beta^e$($\varepsilon$) of the combined (upscaled) component.

\textbf{[3]} Determine $\alpha_v$: 
\begin{align}
\alpha _{v} =\frac{\beta ^{e} \left(\varepsilon \right)-\beta ^{e} }{\varepsilon } \label{0.159)} 
\end{align}

\textbf{[4]} For all random variables $k=1...n$, determine the equivalent $\alpha$-value, $\alpha_k^e$:
\begin{align}
\alpha _{k}^{e} =\sqrt{1-\alpha _{v}^{2} } \frac{\alpha _{k} \sqrt{1-\rho _{k} } }{\sqrt{1-\rho } } +\alpha _{v} \frac{\alpha _{k} \sqrt{\rho _{k} } }{\sqrt{\rho } } \label{0.163)} 
\end{align}

in which $\rho _{k}$ is the correlation between the $k^{th}$ random variable of an element with the corresponding ($k^{th}$) random variable in another element

\textbf{[5]} Normalise the equivalent $\alpha$-values:
\begin{align}
\alpha _{k;final}^{e} =\frac{\alpha _{k}^{e} }{\sqrt{\sum _{j=1}^{n}\left(\alpha _{j}^{e} \right)^{2}  } } \quad ;k=1...n\label{0.122)} 
\end{align}

The equivalent $\alpha$-values of the $n$ variables for the combined $n_{e}$ components are obtained with only two applications of the upscaling method (steps 1 and 2 above).
This is very efficient, when compared to the Hohenbichler method, which needs to be repeated $2n+1$ times to derive the equivalent $\alpha$-values for combining only two components.
Another difference with the Hohenbichler method is that the method for identical components can be applied for non-integer values of $n_{e}$; this is not the case for the Hohenbichler method.
The application with non-integer values of $n_{e}$ is useful for example for upscaling the failure probability of a cross section to the failure probability of a dike section, as will be explained later on in this document.

\paragraph*{Detailed explanation}

Consider the system of $n_e$ identical components as described in equation \eqref{ZEqnNum573383}:
\begin{align}
Z_{i} =\beta -\alpha _{1} U_{i1} -...-\alpha _{n} U_{in} \quad ;i=1...n_{e} \label{ZEqnNum188185} 
\end{align}

These components are combined according to the method as described in \Aref{Section_2.4.3.1}, resulting in a failure probability and associated reliability index $\beta^e$.
The combined component can be described by: % (similar to equation \eqref{ZEqnNum645187}): 
\begin{align}
Z^{e} =\beta ^{e} -\alpha _{1}^{e} U_{1} -...-\alpha _{n}^{e} U_{n} \label{ZEqnNum266010} 
\end{align}

The superscript ''e'' in this equation refers to the fact that these are equivalent values and functions.
In order to derive the $\alpha$-values of function $Z^e$, recall from \Aref{Section_2.2.5} that the $\alpha$-values of a $Z$-function are related to the reliability index $\beta^e$ as follows: 
\begin{align}
\frac{\partial \beta ^{e} }{\partial \bar{u}_{k} } =\alpha _{k}^{e} \quad ;k=1...n\label{0.148)} 
\end{align}

In which $\overline{u}_i$ is the mean of variable $U_i$.
Note the value $\alpha_k^e$ represents the combined effect of variables $U_{1k}$...$U_{n_{e}k}$, and that these variables are mutually correlated.
In order to determine $\alpha_k^e$, these variables need to be split in an independent and mutually dependent part, similar to the description in \Aref{Section_2.4.3.1}.
Consider for this purpose equation \eqref{ZEqnNum188185}.
The different $U$-variables within a single component are mutually uncorrelated, whereas corresponding $U$-values in different components are correlated.
In formula:
\begin{align}
\begin{array}{l} {\rho \left(U_{ij} ,U_{ik} \right)=0\quad   ;j\ne k} \\ {\rho \left(U_{ik} ,U_{\ell k} \right)=\rho _{k} \quad ;i\ne \ell } \end{array}\label{ZEqnNum568983} 
\end{align}

Each $U$-variable can therefore be split in a correlated and uncorrelated part: 
\begin{align}
U_{ik} =U_{ik}^{*} \sqrt{1-\rho _{k} } +V_{k} \sqrt{\rho _{k} } \label{ZEqnNum383869} 
\end{align}

%In which $U_{ik}^{*}$ and  $V_k$ are mutually independent standard normally distributed variables. 
Furthermore, the variables $U_{ik}^{*}$, $i=1...n_e$, $k=1...n$ and  $V_k$, $k=1...n$ are all taken to be mutually independent:
\begin{align}
\begin{array}{l} {\rho \left(U_{ij}^{*} ,U_{k\ell }^{*} \right)=0\quad ;i\ne j\cup k\ne \ell } \\ {\rho \left(U_{ij}^{*} ,V_{k} \right)=0} \\ {\rho \left(V_{j} ,V_{k} \right)=0\quad ;j\ne k} \end{array}\label{ZEqnNum753050} 
\end{align}

With this formulation, variables $U_{ik}$, $i=1...n_e$, $k=1...n$, automatically fulfill requirement \eqref{ZEqnNum568983}, as can be shown in the same manner as was done below equation \eqref{ZEqnNum257919} in the previous section.
Substituting equation \eqref{ZEqnNum383869} in equation \eqref{ZEqnNum188185} gives:
\begin{align}
\begin{array}{l} {Z_{i} =\beta -\alpha _{1} \left(U_{i1}^{*} \sqrt{1-\rho _{1} } +V_{1} \sqrt{\rho _{1} } \right)-...-\alpha _{n} \left(U_{in}^{*} \sqrt{1-\rho _{n} } +V_{n} \sqrt{\rho _{n} } \right)\quad ;i=1...n_{e} } \\ {\quad =\beta -\sum _{k=1}^{n}\alpha _{k}  U_{ik}^{*} \sqrt{1-\rho _{k} } -\sum _{k=1}^{n}\alpha _{k} V_{k}  \sqrt{\rho _{k} } \quad ;i=1...n_{e} } \end{array}\label{0.152)} 
\end{align}

This equation can be replaced by:
\begin{align}
Z_{i} =\beta -U_{i}^{*} \sqrt{1-\rho } -V\sqrt{\rho } \quad ;i=1...n_{e} \label{ZEqnNum395894} 
\end{align}

In which:
\begin{align}
\begin{array}{l} {\rho =\sum _{k=1}^{n}\left(\alpha _{k} \right)^{2}  \rho _{k} } \\ {U_{i}^{*} =\frac{1}{\sqrt{1-\rho } } \sum _{k=1}^{n}\alpha _{k} U_{ik}^{*} \sqrt{1-\rho _{k} }  \quad ;i=1...n_{e} } \\ {V=\frac{1}{\sqrt{\rho } } \sum _{k=1}^{n}\alpha _{k} V_{k}  \sqrt{\rho _{k} } } \end{array}\label{ZEqnNum809022} 
\end{align}

The validity of this replacement can be easily verified by substituting the formulations of $U_i$ and $V$ of equation \eqref{ZEqnNum809022} into equation \eqref{ZEqnNum395894}.
Equation \eqref{ZEqnNum395894} is equivalent to equation \eqref{ZEqnNum883030} if and only if $U_i^{*}$ and $V$ are mutually independent standard normally distributed variables.
The mutual independency can easily be shown since all components $U_{ik}^{*}$, $i=1...n_e$, $k=1...n$, and  $V_k$, $k=1...n$ are mutually independent (see \eqref{ZEqnNum753050}. 

To verify if $U_i^{*}$ and $V$ are standard normally distributed we apply the following general rule (see, e.g. \cite{GrimmettStirzaker1983}): If $X$ and $Y$ are independent normally distributed random variables, then $aX+bY$ is also normally distributed with a mean, $\mu$, and standard deviation, $\sigma$, equal to:
\begin{align}
\begin{array}{l} {\mu =a\mu _{X} +b\mu _{Y} } \\ {\sigma =\sqrt{a^{2} \sigma _{X}^{2} +b^{2} \sigma _{Y}^{2} } } \end{array}\label{0.155)} 
\end{align}

Application of this rule on equation \eqref{ZEqnNum809022}, where all components $U_{ik}^{*}$ and  $V_k$ are normally distributed with mean $0$ and standard deviation $1$, gives:
\begin{align}
\begin{array}{l} {\mu \left(U_{i} \right)=\frac{1}{\sqrt{\rho } } \sum _{k=1}^{n}\alpha _{k} \cdot 0\cdot \sqrt{1-\rho _{k} }  =0} \\ {\sigma \left(U_{i} \right)=\sqrt{\left(\frac{1}{\sqrt{1-\rho } } \right)^{2} \sum _{k=1}^{n}\left(\alpha _{k} \right)^{2} \cdot \left(1\right)^{2} \cdot \left(1-\rho _{k} \right) } =\sqrt{\frac{1}{1-\rho } \left[\sum _{k=1}^{n}\alpha _{k}^{2}  -\sum _{k=1}^{n}\alpha _{k}^{2} \rho _{k}  \right]} } \\ {\quad \quad   =\sqrt{\frac{1}{1-\rho } \left[1-\rho \right]} =\sqrt{1} =1} \end{array}\label{0.156)} 
\end{align}

This shows that $U_i^{*}$ and $V$ in equation \eqref{ZEqnNum395894} are mutually independent standard normally distributed variables.
Taking into account the formulation of the $Z$-function in equation \eqref{ZEqnNum395894}, The equivalent coefficient $\alpha_k^e$ can now be derived as follows
\begin{align}
\alpha _{k}^{e} =\frac{\partial \beta ^{e} }{\partial \bar{u}_{k} } =\frac{\partial \beta ^{e} }{\partial \bar{u}} \frac{\partial \bar{u}}{\partial \bar{u}_{k} } +\frac{\partial \beta ^{e} }{\partial \bar{v}} \frac{\partial \bar{v}}{\partial \bar{u}_{k} } \label{ZEqnNum438636} 
\end{align}
where $\bar{u}$, $\bar{v}$ and $\bar{u}_k$ are the mean values of variables $U^{*}$, $V$ and $U_k$.
So, the derivation of coefficients $\alpha_k$ $i=1...n_e$ it comes down now to determining the four partial derivatives of equation \eqref{ZEqnNum438636}.
The first two can be determined directly from equation \eqref{ZEqnNum809022}:
\begin{align}
\begin{array}{l} {\frac{\partial \bar{u}}{\partial \bar{u}_{k} } =\frac{\alpha _{k} \sqrt{1-\rho _{k} } }{\sqrt{1-\rho } } } \\ {\frac{\partial \bar{v}}{\partial \bar{u}_{k} } =\frac{\alpha _{k} \sqrt{\rho _{k} } }{\sqrt{\rho } } } \end{array}\label{ZEqnNum945170} 
\end{align}

The partial derivative of $\beta^e$ to $\bar{v}$ is determined numerically:
\begin{align}
\alpha _{v} =\frac{\partial \beta ^{e} }{\partial \bar{v}} \approx \frac{\beta ^{e} \left(\varepsilon \right)-\beta ^{e} }{\varepsilon } \label{ZEqnNum432286} 
\end{align}

In which $\beta^e$($\varepsilon$) is the reliability index of the upscaled system of $n_e$ components, after perturbation of $\bar{v}$ with a small value $\varepsilon$. 
\begin{align}
\beta ^{e} \left(\varepsilon \right)=\Phi ^{-1} \left(1-P\left(Z^{e} \left(\varepsilon \right)<0\right)\right)=\Phi ^{-1} \left(1-\bigcup _{i=1}^{n_{e} }P\left(Z_{i} \left(\varepsilon \right)\right) \right)\label{ZEqnNum242869} 
\end{align}

In which function $Z_i$($\varepsilon$) is as follows: 
\begin{align}
\begin{array}{l} {Z_{i} \left(\varepsilon \right)=\beta -U_{i}^{*} \sqrt{1-\rho } -\left(V+\varepsilon \right)\sqrt{\rho } \quad ;i=1...n_{e} } \\ {\quad \quad    =\left(\beta -\sqrt{\rho } \varepsilon \right)-U_{i}^{*} \sqrt{1-\rho } -V\sqrt{\rho } \quad ;i=1...n_{e} } \end{array}\label{ZEqnNum791362} 
\end{align}

In other words: $Z_i$($\varepsilon$) is a $Z$-function with a reliability index $\beta_{Z_{\varepsilon}}$ following:
\begin{equation}
%\beta_Z = \frac{\mu_Z}{\sigma_Z} = \beta - \frac{\epsilon v}{\sqrt{\rho}}
\beta_{Z_{\varepsilon}} = \frac{\mu_Z}{\sigma_Z} = \frac{\beta  - \sqrt \rho  \varepsilon }{1} = \beta  - \sqrt \rho  \varepsilon 
%\{\beta _{{Z_\varepsilon }}} = \frac{{{\mu _{{Z_\varepsilon }}}}}{{{\sigma _{{Z_\varepsilon }}}}} = \frac{{\beta  - \sqrt \rho  \varepsilon }}{1} = \beta  - \sqrt \rho  \varepsilon \
\end{equation}
So $\beta^e$($\varepsilon$) is quantified by substituting equation \eqref{ZEqnNum791362} into equation \eqref{ZEqnNum242869} and subsequent application of the upscaling procedure of \Aref{Section_2.4.3.1}.
Subsequently, $\beta^e$($\varepsilon$) is substituted in equation \eqref{ZEqnNum432286} in order to derive $\alpha_v$ the partial derivative of $\beta^e$ to $\bar{v}$.
The next step is to derive the partial derivative of $\beta^e$ to $\bar{u}$.
This can be derived as follows:
\begin{align}
\frac{\partial \beta ^{e} }{\partial \bar{u}} =\sqrt{1-\left(\frac{\partial \beta ^{e} }{\partial \bar{v}} \right)^{2} } =\sqrt{1-\alpha _{v}^{2} } \label{ZEqnNum165859} 
\end{align}

This can be explained as follows: the partial derivative of $\beta^e$ to $\bar{v}$ is the resulting $\alpha$-value for the dependent part of the $n_e$ components, represented by variable $V$.
The partial derivative of $\beta^e$ to $\bar{u}$ is the resulting $\alpha$-value for the independent part of the $n_e$ components, represented by variable $U^{*}$.
The sum of the squares of these $\alpha$-values should be equal to $1$. 

Substitution of equations \eqref{ZEqnNum945170}, \eqref{ZEqnNum432286} and \eqref{ZEqnNum165859} into equation \eqref{ZEqnNum438636} provides the requested equivalent $\alpha$-values:
\begin{align}
\alpha _{k}^{e} =\frac{\partial \beta ^{e} }{\partial \bar{u}_{k} } =\sqrt{1-\alpha _{v}^{2} } \frac{\alpha _{k} \sqrt{1-\rho _{k} } }{\sqrt{1-\rho } } +\alpha _{v} \frac{\alpha _{k} \sqrt{\rho _{k} } }{\sqrt{\rho } } \label{ZEqnNum998801} 
\end{align}

The $Z$-function of the resulting component from the upscaling procedure (equation \eqref{ZEqnNum266010}) needs to have a standard deviation equal to $1$.
This means the sum of the squares of the equivalent $\alpha$-values shoud be equal to $1$.
Equation \eqref{ZEqnNum998801} guarantees that this is the case if all values of $\rho_k$ are equal to either $0$ or $1$, which is generally the case for upscaling in time (i.e.\ slow varying random load variables and strength variables have an autocorrelation equal to $1$, while fast varying random variables have an autocorrelation equal to 0).
This can be deducted as follows: 
\begin{align}
\begin{array}{l} {\sum _{k=1}^{n}\left(\alpha _{k}^{e} \right)^{2}  =\sum _{k=1}^{n}\left[\left(1-\alpha _{v}^{2} \right)\frac{\alpha _{k}^{2} \left(1-\rho _{k} \right)}{1-\rho } +2\alpha _{v} \left(\sqrt{1-\alpha _{v}^{2} } \right)\frac{\alpha _{k} \sqrt{1-\rho _{k} } }{\sqrt{1-\rho } } \frac{\alpha _{k} \sqrt{\rho _{k} } }{\sqrt{\rho } } +\alpha _{v}^{2} \frac{\alpha _{k}^{2} \rho _{k} }{\rho } \right] } \\ {\quad \quad \quad    =\sum _{k=1}^{n}\left[\left(1-\alpha _{v}^{2} \right)\frac{\alpha _{k}^{2} \left(1-\rho _{k} \right)}{1-\rho } +\alpha _{v}^{2} \frac{\alpha _{k}^{2} \rho _{k} }{\rho } \right] } \\ {\quad \quad \quad    =\frac{\left(1-\alpha _{v}^{2} \right)}{1-\rho } \sum _{k=1}^{n}\alpha _{k}^{2} \left(1-\rho _{k} \right)+ \frac{\alpha _{v}^{2} }{\rho } \sum _{k=1}^{n}\alpha _{k}^{2} \rho _{k}  } \\ {\quad \quad \quad    =\frac{\left(1-\alpha _{v}^{2} \right)}{1-\rho } \left(1-\rho \right)+\frac{\alpha _{v}^{2} }{\rho } \rho =\left(1-\alpha _{v}^{2} \right)+\alpha _{v}^{2} =1} \end{array}\label{0.164)} 
\end{align}

\Note{in the second step of this equation, the middle term is removed because it is equal to zero (since either $\rho_k=0$ or $1-\rho_k = 0$).
In the fourth step, equation \eqref{ZEqnNum809022} is used.
If not all values of $\rho_k$ are equal to either $0$ or $1$, the sum of the squares of the equivalent $\alpha$-values is not necessarily equal to $1$.
In that case, they have to be normalized as follows:}
\begin{align}
\alpha _{k}^{e} =\frac{\alpha _{k}^{e} }{\sqrt{\sum _{j=1}^{n}\left(\alpha _{j}^{e} \right)^{2}  } } ;k=1...n\label{0.165)} 
\end{align}

\section{Techniques for time and space dependent processes}\label{Section_2.4.5}
In this section, aspects of upscaling in time and space are addressed.

The techniques described in the previous sections all deal with system analysis of a discrete number of components which may represent dike sections, wind directions, etc.
In some applications, however, $Z$ is a function of space and time, which means in principle the number of components is infinite.
This is schematically depicted in \Fref{fig:2.32}.
In the left panel, $Z$ is a time-dependent function and failure potentially can occur at any time.
On the right, $Z$ is a function of space, and failure can occur at any location.

\begin{figure}[H]\centering
\includegraphics*[width=5.93in, height=1.58in, keepaspectratio=false]{\pathProbLibPictures variation_in_time_and_space}
\caption{Stochastic variation of the $Z$-function in time (left) and space (right)}\label{fig:2.32}
\end{figure}

This section describes some approaches to deal with these type of continuos descriptions of $Z$-functions.

\subsection{Poisson counting process} \label{Section_2.4.5.2}
The Poisson counting process describes the probability of occurrence of $n$ events, where a single event generally refers to an upcrossing or downcrossing of a threshold value.
With respect to failure, the downcrossing of the threshold $Z=0$ is most relevant.
In a Poisson process it is assumed that for small values of $\Delta t$ [a] the occurrence of an event in an interval $[t, t+\Delta t]$ is proportional to $\Delta t$ and [b] the probability of occurrence of two events occurring in $[t, t+\Delta t]$ is negligible.
This means for small values of $\Delta t$, the probability of an event occurring in $[t, t+\Delta t]$ is approximately equal to:
\begin{align}
P\left(\mbox{1 event during } \left[t,t+\Delta t\right]\right)\approx \upsilon \Delta t\label{ZEqnNum344717} 
\end{align}

In this equation, $\nu$ is the `intensity' of the Poisson process.
This is the single parameter that describes the Poisson process.
Define $N(t)$ as the number of events occurring in the time interval  $[0,T]$.
For a Poisson process the probability distribution of $N(t)$ is:
\begin{align}
P\left(N(t)=n\right)=\frac{\left(\nu t\right)^{n} e^{-\nu t} }{n!} \label{ZEqnNum870173} 
\end{align}

The time interval between two subsequent events is also a random variable and it is exponentially distributed.
So if $t_{1}$ is the time interval between two events, then:
\begin{align}
P\left(T_{1} \le t_{1} \right)=1-e^{-\nu t_{1} } \label{ZEqnNum865643} 
\end{align}

The assumption of a Poisson process is often used to translate exceedance frequencies into exceedance probabilities.
Suppose $\nu$ is expressed as ''number of events per year''.
In that case $\nu$ is the annual frequency of occurrence.
Then, according to equation \eqref{ZEqnNum865643}, the annual probability of occurrence is equal to: 
\begin{align}
P\left(T_{1} \le 1\right)=1-e^{-\nu } \label{ZEqnNum774553} 
\end{align}

This shows the relation between probability and frequency in case of a Poisson process.
An event can be for instance the exceedance of a threshold level $x$, for load variable $X$.
In that case, equation \eqref{ZEqnNum774553} can be applied to translate the annual frequency of exceedance of threshold $x$ into the annual probability of exceedance of threshold $x$, or vice versa.

In the description above, $\nu$ was assumed to be time-independent.
If this is not the case, equation \eqref{ZEqnNum870173} changes into:
\begin{align}
P\left(N(t)=n\right)=\frac{\left(\int _{0}^{t}\nu \left(\tau \right)d\tau  \right)^{n} e^{-\int _{0}^{t}\nu \left(\tau \right)d\tau  } }{n!} \label{0.171)} 
\end{align}

\subsection{Outcrossing}\label{Section_2.4.5.3} 

If an event refers to failure in a continuous process, i.e.\ the downcrossing of threshold $Z=0$ in \Fref{fig:2.32}, then the outcrossing rate is defined as:
\begin{align}
v=\begin{array}{c} {\lim } \\ {\Delta t\downarrow 0} \end{array}  \frac{P\left[Z\left(t\right)\ge 0\cap Z\left(t+\Delta t\right)<0\right]}{\Delta t} \label{ZEqnNum351501} 
\end{align}

Note that the numerator in this equation is the probability that failure occurs in time interval $[t,t+\Delta t]$.
The rate $\nu$ is similar to the one defined in the previous section and can also be time-dependent: $\nu$=$\nu(t)$.
Assume for the moment that $\nu$ is a constant, i.e.\ independent of time.
The probability that failure occurs in an interval  $(0,T]$, given the fact that no failure occurs at  $t=0$, is then equal to:
\begin{align}
P\left[\begin{array}{c} {\min } \\ {t\in \left(0,T\right]} \end{array} \left\{Z(t)\right\}<0\left| Z\left(0\right)\ge 0\right. \right]=1-e^{-vT} \label{0.173)} 
\end{align}

Note that this probability of failure is described by an exponential distribution function.
The exponential distribution is by definition the distribution which describes failure probabilities for processes with a constant failure rate (see e.g. \cite{GrimmettStirzaker1983}).
\Fref{fig:2.33} shows an example of an exponential distribution function.
In this figure the failure rate, $\nu$, is taken equal to $1$, which makes this a standard exponential distribution function. 

\begin{figure}[H]\centering
\includegraphics*[width=4.45in, height=3.34in, keepaspectratio=false]{\pathProbLibPictures exponentional_distribution}
\caption{Standard exponential distribution function:  $F(T) = 1-exp(-T)$.}\label{fig:2.33}
\end{figure}

The probability that no failure occurs in an interval  $(0,T]$, i.e.\  $t=0$ included, given the fact that no failure occurs at  $t=0$, is equal to:
\begin{align}
P\left[\begin{array}{c} {\min } \\ {t\in \left(0,T\right]} \end{array} \left\{Z(t)\right\}\ge 0\left| Z\left(0\right)\ge 0\right. \right]=e^{-vT} \label{0.174)} 
\end{align}

The probability that no failure occurs in an interval  $[0,T]$, i.e.\  $t=0$ included, is then equal to:
\begin{align}
P\left[\begin{array}{c} {\min } \\ {t\in \left[0,T\right]} \end{array} \left\{Z(t)\right\}\ge 0\right]=\left[1-P_{{\rm F}} \left(0\right)\right]e^{-vT} \label{0.175)} 
\end{align}

In which  $P_F(0)$ is the initial probability of failure (see below for more information on this probability), $i$.e the probability that $Z<0$ at  $t=0$.
The probability, $P_f$, that failure occurs in an interval  $[0,T]$ is equal to:
\begin{align}
P_{F} \left(T\right)=P\left[\begin{array}{c} {\min } \\ {t\in \left[0,T\right]} \end{array} \left\{Z(t)\right\}<0\right]=1-{\rm  }\left[1-P_{{\rm F}} \left(0\right)\right]e^{-vT} \label{ZEqnNum473284} 
\end{align}

If the outcrossing rate, $\nu$, and the initial failure probability,  $P_F(0)$, are small, the probability of failure can be approximated by:
\begin{align}
P_{F} \left(T\right)\approx P_{{\rm F}} \left(0\right)+\nu T\label{ZEqnNum950209} 
\end{align}

This is an upper bound of the failure probability.
In essence, this approximation ''double counts'' the probability of events in which two or more failures occur in the interval  $[0,T]$.
If $\nu$ and  $P_F(0)$ are small, the probability of two or more failures occurring in the interval  $[0,T]$ is negligible and therefore equation \eqref{ZEqnNum950209} is a good approximation in that case.

In the equations above, failure rate $\nu$ was assumed to be constant.
If this is not the case, equation \eqref{ZEqnNum473284} changes into the following, more generic, equation: 
\begin{align}
P_{F} \left(T\right)=1-{\rm  }\left[1-P_{{\rm F}} \left(0\right)\right]\exp \left(-\int _{0}^{T}v\left(t\right)dt \right)\label{ZEqnNum131880} 
\end{align}

The equations above can also be used if $Z$ is a function of space.
In that case, $t$ and $T$ need to be replaced by $x$ and $X$, where $x$ represents distance, e.g. the longitudinal distance along a dike section. 

\Note{in \probLib, the outcrossing method is applied in time as well as in space.
First, the probabilities of failure of the smallest ''components'' are computed with the probabilistic techniques for single components as described in \Aref{chp:ReliabilityAnalysis}.
The smallest component is e.g. a cross section of a flood defense (space) during a tidal period (time) for a single failure mechanism.
So, the initial result of the probabilistic procedure is the probability that failure occurs at a certain cross section within the time-span of a tidal period for a single mechanism.
This result will be used as  $P_F(0)$ in the equations above, i.e. the initial failure probability.
Subsequently the outcrossing approach is applied for upscaling the probability of failure (for the mechanism under consideration) from a cross-section to a dike section and from a tidal period to a year.

The failure rate $\nu(t)$ or $\nu($x$)$ needs to be derived from spatial and temporal autocorrelations of the strength and load variables.
This is described in more detail in \cite{TechRef}.
In general, functions $\nu(t)$ and $\nu(x)$ are too complex to solve equation \eqref{ZEqnNum131880} analytically, which means approximating techniques are required.
The \probLib uses different outcrossing approaches for space and time because of mutual differences in autocorrelation structures.

Note that the component for which  $P_F(0)$ is computed in \probLib has a ''width'' equal to the assumed breach width.
This means a (slight) reduction in the length of the remainder of the dike section and hence a (slight) reduction in the computed failure probability.
The assumed breach width depends on the mechanism under consideration.
The ''width'' in time is taken equal to a tidal period.
This has to do with the fact that the input statistics of random load variables like sea water level, river discharge or wind speed represent probabilities of the maximum value in a tidal period (see also \cite{TechRef}).
These values are therefore suitable to represent the whole tidal period.}

\section{Spatial upscaling - from cross section to flood defence segment}\label{Section_2.5.5}
In this section, upscaling in space is discussed.
This type of upscaling is applicable to flood defence systems.

\subsection{Computing the failure probability}\label{Section_2.5.5.1}
The spatial upscaling technique as described in the current section is done over homogeneous reaches of the flood defense.
Homogenous in this case means the statistical characteristics remain constant.
It is therefore relevant that the flood defence system is divided into segments for which the assumption of homogeneity is valid.
So, if a dike segment is inhomogeneous, it needs to be split up into smaller, homogenous, segments.

Spatial upscaling is subject to a concept known as the length effect.
The length effect essentially has to do with the increase in failure probability when going from a cross-section to a longitudinal segment and from a single segment to a flood defense system (interconnected segments).
That is, the length effect refers to the effect that an increase in length has on the probability of failure.
Note that this effect is also present when upscaling over time; the failure probability will increase as the considered time period increases. 

The mathematical description of the length effect is the ratio of the failure probability of the larger length to that of the shorter.
For the upscaling from cross-section to longitudinal segment (assuming statistical homogeneity!) this would be as follows: 
\begin{align}
\text{Length effect} = \frac{P_{f, segment} }{P_{f, cross-section} } \label{ZEqnNum163660} 
\end{align}
where $P_{f,segment}$ refers to the failure probability of the longitudinal segment and $P_{f,cross-section}$ refers to the failure probability of the cross section within that longitudinal segment.
To derive the ratio of equation \eqref{ZEqnNum163660}, a notion of the spatial correlation within the segment is required, for each random variable, $X$, involved.
In \probLib this correlation is described with the following model:
\begin{align}
\rho \left(\Delta y\right)=\rho _{x} +(1-\rho _{x} )\exp \left[-\left(\frac{\Delta y^{2} }{d_{x}^{2} } \right)\right]\label{ZEqnNum807806} 
\end{align}
where $\rho$ is the correlation between two locations within the segment ($\rho \geqslant 0$), $\Delta y$ is the distance between these two locations, $\rho_x$ is the residual correlation length of variable $X$ and $d_x$ is the spatial correlation length of variable $X$.
Parameter $d_x$ determines how quickly the correlation of variable $X$ decreases over distance and $\rho_x$ is the minimum correlation of variable $X$ between two locations of the same (homogeneous) segment.
The parameters $d_x$ and $\rho_x$ need to be determined for each variable $X$, based on a combination of measurements and expert judgement.

\begin{figure}[H]\centering
\includegraphics*[width=4.09in, height=3.06in, keepaspectratio=false]{\pathProbLibPictures autocorrelation_function}
\caption{Autocorrelation function, correlation within a dike section; in this picture, the correlation $\rho$ is visualized against $\Delta x$, made non-dimensional by the correlation distance $d_x$.}\label{fig:2.40}
\end{figure}

The correlation model of equation \eqref{ZEqnNum807806} and \Fref{fig:2.40} in principle is applied for each strength variable (load variables can generally be assumed to have correlation $1$ within a single segment).
This results in a similar model for the $Z$-function, i.e.\ in values $d_Z$ and $\rho_Z$:
\begin{align}
\rho \left(\Delta y\right)\approx \rho _{Z} +(1-\rho _{Z} )\exp \left[-\left(\frac{\Delta y^{2} }{d_{Z}^{2} } \right)\right]  \label{eq:rhodely} 
\end{align}

The parameters $d_Z$ and $\rho_Z$ can be derived as follows: 
\begin{align}
\rho _{Z} =\sum _{i=1}^{n}\alpha _{i}^{2} \rho _{i}  \label{eq:sm2} 
\end{align}
\begin{align}
\frac{1}{d_{Z}^{2} } =\frac{1}{1-\rho _{Z} } \sum _{i=1}^{n}\alpha _{i}^{2} (1-\rho _{i} )\frac{1}{d_{i}^{2} }  \label{ZEqnNum304048} 
\end{align}

In which: 

\begin{tabular}{p{\textwidth-36pt-125mm}p{120mm}}
$d_i$ & is the correlation length of random variable $i$ \\
$\rho_i$& is the residual correlation length of random variable $i$ \\
$\alpha_i$& is the influence coefficient of random variable $i$ \\
\end{tabular}

Note that coefficients $\alpha_1$, \dots ,$\alpha_n$ are determined in the probabilistic computation for a ''representative'' cross-section within the flood defence segment.
For this purpose the probabilistic techniques for a single component are used (see \autoref{chp:ReliabilityAnalysis}).

To derive the probability of failure of a dike segment, the segment is divided into components of equal length $\Delta L$.
The number of components is equal to:
\begin{align}
n_{e} =\frac{L}{\Delta L} 
\end{align}
where $L$ is the length of the dike segment.
The probability of failure for the entire dike segment is then equal to:
\begin{align}
P_{f,segment} \approx \left(1+n_{e} \right)P_{f,cross-section} =\left(1+\frac{L}{\Delta L} \right)P_{f,cross-section} \label{ZEqnNum665818} 
\end{align}

This means the continuous process, in which failure can occur at any location along the dike is now replaced by a discrete process in which the dike segment is composed of a finite number of components, each of which has a failure probability that is equal to the probability of failure of a cross-section.
This simplification/approximation is only valid for a well selected value of $\Delta L$.
If we assume that the spatial variation of $Z$ is a Gaussian ergodic process (i.e. $\rho_Z=0$), the length $\Delta L$ should be taken equal to:
\begin{align}
\Delta L=d_{Z} \sqrt{\pi } /\beta \quad ;{\rm if} ~ \rho _{Z} =0\label{ZEqnNum979631} 
\end{align}
where $\beta$ is the reliability index as derived in the probabilistic computation for a cross-section (see \autoref{chp:ReliabilityAnalysis}).
The value of $\Delta L$ is a result of the outcrossing approach (see \Aref{Section_2.4.5.3}) in which the spatial variation of $Z$ is assumed to be a Gaussian ergodic process.
The derivation of $\Delta L$, as described in equation \eqref{ZEqnNum979631}, is described in \cite{Jongejan2012}.

With the assumption of a Gaussian ergodic process, the failure probability of a dike segment of length $L$ is approximately equal to (combine equations \eqref{ZEqnNum665818} and \eqref{ZEqnNum979631}):
\begin{align}
P_{f,segment} \approx \left(1+\frac{L\beta }{d_{Z} \sqrt{\pi } } \right) \Phi \left(-\beta \right)\quad ;{\rm if} ~ \rho _{Z} =0\label{eq:sm3} 
\end{align}

If $\rho_Z$$>$0, the assumption of a Gausian ergodic process does not hold and an alternative solution is required.
In that case, $\rho_Z>0$ represents the part of the correlation function that does not contribute to the length effect, because it is the correlation that persists over the entire dike segment.
In that case the $Z$-function is split in an ergodic part (with $\rho$ approaching zero over long distances) and a non ergodic part (with $\rho$ constant):
\begin{align}
Z=\beta -v\sqrt{\rho } -u\sqrt{1-\rho } \label{ZEqnNum628933} 
\end{align}
where $v$ is the non-ergodic constant and $u$ is the ergodic stochastic process with:
\begin{align}
\rho \left(\Delta y\right)=\exp \left[-\left(\frac{\Delta y^{2} }{d_{Z}^{2} } \right)\right] \label{eq:sm4} 
\end{align}
where $\rho$ is the correlation between two locations within the segment and $\Delta y$ is the distance between two locations.
Using the theorem of total probability, the failure probability of the flood defense segment can be described as follows:
\begin{align}
P\left[Z<0\right]=\int _{}^{}P\left[Z<0|v\right]f_{V} \left(v\right)dv \label{ZEqnNum551150} 
\end{align}
where $f_V(v)$ is the standard normal density function.
The conditional failure probability, $P[Z<0|v]$, in equation \eqref{ZEqnNum551150} can be written as (see Jongejan, 2012):
\begin{align}
\begin{array}{l} {P\left[Z<0|v\right]=1-\left(1-P\left(Z_{cross} <0\right)\right)e^{-N_{f} } } \\\\ {N_{f} = \displaystyle \frac{L}{2\pi } e^{-\frac{\beta^{*2} }{2} } \frac{\sqrt{2} }{d_{z} } } \\\\ {\beta^{*}= \displaystyle \frac{\beta _{cross} -v\sqrt{\rho _{z} } }{\sqrt{1-\rho _{z} } } } \end{array}\label{ZEqnNum591781} 
\end{align}
where $Z_{cross}$ and $\beta_{cross}$ are the $Z$-function and reliability index of the cross section and $\Phi$ is the standard normal distribution function.
The combination of equations \eqref{ZEqnNum551150} and \eqref{ZEqnNum591781} provide the probability of failure for a flood defense segment.
More details on the derivation of equations \eqref{ZEqnNum551150} and \eqref{ZEqnNum591781} can be found in \cite{Jongejan2012}.
Equation \eqref{ZEqnNum591781} can be evaluated with high accuracy using for example numerical integration.

\Note{in formula \ref{eq:sm3}  the width of the mechanism is not taken into account.
In \probLib the width of the mechanism is taken equal to $\Delta L$.
In the formula for $N_F$ the length $L$ is replaced by $L$ - $\Delta L$.
The idea behind this correction is that for stretches smaller than $\Delta L$ it is not be possible to have an increase in failure probability as a result of the length effect.}

%In earlier versions of PC-Ring, the following approximation for equations \eqref{ZEqnNum551150} and \eqref{ZEqnNum591781} was implemented to save computation time:
%\begin{align}
%P\left[Z<0\right]=\left(1+\frac{L\beta \sqrt{1-\rho _{z} } }{d_{Z} \sqrt{\pi } } \right)\Phi \left(-\beta %\right)\label{ZEqnNum246851}
%\end{align}

%This approximation is only valid for small values of $\rho_z$.
%With the current day computation power, equation \eqref{ZEqnNum591781} can be evaluated in a split second, so it is recommended not to use the approximation as described with equation \eqref{ZEqnNum246851}.

\Note{formula \eqref{ZEqnNum591781} is only valid for value of $\rho_z$ $>$ $0$.
For values of $\rho_z = 0$ the Hohenbichler method is used in \probLib in combination with the outcrossing approach.}

\subsection{Computing equivalent $\alpha$-values}\label{Section_2.5.5.2}

As stated in the previous section, the flood defence segment can be thought of to consist of identical components of $n$ identical components of length $\Delta L$.
Upscaling to a dike section in essence is therefore the same as upscaling over $n$ identical components.
The last step in such an upscaling process, is the derivation of new equivalent $\alpha$-values for the individual random variables, see \Aref{Section_2.4.3.2}.
The first step in this method is to determine the $\alpha$-value of the correlated part of the $Z$-function of equation \eqref{ZEqnNum628933}, i.e.\ variable $V$.
This is done in the standard way by perturbing the mean value of $V$ with a small value $\varepsilon$ and quantifying the effect on the computed $\beta$-value of the dike section of a small perturbation ($\varepsilon$) in the mean value of $V$.
The $\alpha$-value of $V$ is thus equal to
\begin{align}
\alpha _{v} =\frac{\partial \beta _{section} }{\partial \bar{v}} 
\end{align}

Equation \eqref{ZEqnNum998801} states that the equivalent value, $\alpha_k^e$, of variable $k$ can then be derived as follows: 
\begin{align}
\alpha _{k}^{e} =\sqrt{1-\alpha _{v}^{2} } \frac{\alpha _{k} \sqrt{1-\rho _{k} } }{\sqrt{1-\rho_Z} }+\alpha _{v} \frac{\alpha _{k} \sqrt{\rho _{k} } }{\sqrt{\rho_Z} } \label{ZEqnNum642181} 
\end{align}

In which $\alpha_k$ is the $\alpha$-value of variable $k$ before upscaling and $\rho_k$ is the correlation of variable $k$ between two components.
Since components in this case have length $\Delta L$, this correlation is equal to (see equation \eqref{ZEqnNum807806}:
\begin{align}
\rho _{k} =\rho _{kr} +\left(1-\rho _{kr} \right)\cdot \exp \left[-\left(\frac{\Delta L}{d_{k} } \right)^{2} \right]\label{eq:sm5} 
\end{align}
where $\rho_{kr}$ is the residual correlation length of variable $k$ and $d_x$ is the spatial correlation length of variable $k$. 
