\chapter{Response surfaces}
\label{sec:ResponseSurface}

A response surface predicts a response value of a model. The advantage of using response surfaces is that it needs less calculation time than a model.

A response surface is trained with a number of of model realizations. Based on these model realizations, all values can be predicted with more or less accuracy. 

It is useful to update the response surface during the calculation. During the calculation it is known which realization to use for recalculation, because of the distance to the limit state.

Response surfaces are defined in x-space or u-space. The advantage of x-space is that response surfaces do not need new model runs when variable distributions change.

\section{Initialize response surface}
\label{sec:ProxyInitialization}

To build response surfaces, a number of model runs have to be performed. Depending on the required usage of the response surface and the length of the computation time, the following options exist:

\begin{itemize}
	\item Single: All stochastic variables are disturbed one by one, while keeping the other stochastic variables at their median value (Q=0.5). The disturbance is in one direction at the Q=0.9 value;
	\item Double: All stochastic variables are disturbed one by one, while keeping the other stochastic variables at their median value (Q=0.5). The disturbance is in two directions at the Q=0.1 and Q=0.9 value;
	\item Experimental design: A smart subset of full factorial combination is generated. The subset should contain samples which are all "different enough" from each other. Experimental design resolution 3 is used.
	\item Full factorial: Each combination of parameters, being either Q=0.1 or Q=0.9, is used. This set is extended with the all Q=0.5 values set. This leads to $2^n + 1$ combinations
	\item Random: The training values are selected random. Optionally a factor can be applied by which the u-values of the realizations are multiplied.
\end{itemize}

\section{Response surface algorithms}
\label{sec:ProxyFunction}

Different types of response surfaces exist. The following response surfaces are supported in the \probLib.

\subsection{Polynome}

The response value $R_r$ is written as

\begin{align}
R_r = a_r + \sum\limits_{i=1}^n b_{r,i} x_i + \sum\limits_{i=1}^n c_{r,i} x_i^2 + \sum\limits_{i,j=1, i\neq j}^{n,n} d_{r,i,j} x_i x_j
\end{align}

where
\begin{longtable*}{p{20mm}p{\textwidth-24pt-20mm}}  
	$r$ & is the index of the response value; \\
	$i$, $j$ & are indices of the stochastic variables; \\
	$x$ & is the sample generated by the probabilistic method; \\  
	$a$, $b$, $c$, $d$ & are coefficients of the response surface function; 
\end{longtable*}

With least squares regression, the response surface coefficients are calculated from the model results. When there are more response surface coefficients than model runs, the cross terms ($d$) and quadratic coefficients ($c$) are omitted. When later more model runs become available (see \autoref{sec:UpdateProxy}), these coefficients are added.

\subsubsection{Update polynome response surface}
\label{sec:UpdateProxy}

During the calculation, samples are selected from which the real model result is requested. This depends on the nature of the probabilistic technique. When such a sample is calculated, its response values are used to update the response surface coefficients.

The update of the response surface coefficients does not use all available samples, but only the ones closest to the design point found so far. The distance $D$ from a sample run to the design point is calculated in u-space by

\begin{align}
\label{eq:ProxyDistance}
D^2 = \sum\nolimits_{i=1}^n \left(u_{i\text{, design}} u_{i\text{, sample}}\right)^2
\end{align}

Then the $n$ samples with lowest distances $D$ are used to update the response surface coefficients, where $n$ is

\begin{align}
\label{eq:ProxyNumberOfRuns}
n = f_{\text{overdetermination}} \cdot n_{\text{required}} 
\end{align}

where 

\begin{longtable*}{p{20mm}p{\textwidth-24pt-20mm}}  
	$f_{\text{overdetermination}}$ & is the user supplied overdetermination factor ($\geq 1$). This value should not be too high, because then all realizations far from the interesting are used and disturb proper predictions. The value should not be too low, because some noise could disturb proper predictions. Recommended value is 1.1; \\
	$n_{\text{required}}$ & is the number of coefficients (including quadratic and cross terms); 
\end{longtable*}

The user can limit the number of response surface updates.

\paragraph{Updates in Monte Carlo methods}
\label{sec:ProxyUpdateMonteCarlo}

This section applies to Crude Monte Carlo and Importance Sampling.

After each calculation of a sample using the response surface, it is evaluated whether it should be recalculated with the model. This will be done if the limit state value indicates failure and the sample represents a reliability index, which is not very unlikely, when compared with the design point found so far. The latter condition is true if

\begin{align}
\label{eq:ProxyMCCondition}
\beta_{\text{sample}} \leq \beta_{\text{design}} + \Delta \beta
\end{align}

where 

\begin{align}
\label{eq:ProxyBeta}
\beta^2 = \sum\nolimits_{i=1}^n u_{i}^2
\end{align}

and 

\begin{longtable*}{p{20mm}p{\textwidth-24pt-20mm}}  
	$\Delta \beta$ & is a user supplied indication when samples should be recalculated (also known as beta-sphere);
\end{longtable*}

When samples meeting this condition are found, the calculation is stopped and the response surface is updated. Then the Monte Carlo calculation starts again. If there was no design point yet, because no failing sample was found so far, no response surface updates take place.

\paragraph{Updates in Directional Sampling}

Response surface updates in directional sampling are similar to response surface updates in Monte Carlo methods (see \autoref{sec:ProxyUpdateMonteCarlo}), but differ because the design point can be updated after a direction has been calculated, instead of a sample. After each direction it is evaluated whether a recalculation with the model is needed, if (similar to \autoref{eq:ProxyMCCondition}) 

\begin{align}
\label{eq:ProxyDSCondition}
\beta_{\text{direction}} \leq \beta_{\text{design}} + \Delta \beta
\end{align}

If this is true, the direction is recalculated with the model. Then the calculation is stopped and the response surface is updated. Then the calculation is started again. Next calculations, which will use exactly the same samples, will reuse prior directions which were recalculated with the model.

\paragraph{Updates in FORM}

When the design point is found using the response surface, the design point is calculated with the model. If \autoref{eq:FORMConvergence} is still true, the calculation ends. If not, the response surface is updated and a new FORM calculation starts.



\subsection{Linear grid}

The response surface is defined as a number of points on a grid. When a value is requested, a linear interpolation takes place between the nearest points on the grid. The grid is not necessarily equally spaced.

When a value is requested beyond the grid, the point closest to the grid is used.

The linear grid response surface can not be updated during the calculation.

\subsection{Gaussian Process Regression}
\label{sec:GPR}

Gaussian Process Regression is used as a model to simulate the original model results. The Gaussian Process Regression is trained with a number of model runs. Based on these runs, a model result can be predicted including uncertainty estimation, see \author{fig:Kriging}.

\begin{figure}[H]
	\label{fig:Kriging}
	\centering
	\includegraphics[scale=0.8]{pictures/Kriging.jpg}
	\caption{Gaussian Process Regression}
\end{figure}

When a value is requested at a point which coincides with one of the training runs, the uncertainty will be zero. A value in between will have a non zero uncertainty. The value will be an interpolation between the trained points, taking into account the distance to these points. The distance is not the Euclidean distance, but calculated with the kernel function. The option Matern 5/2 will give the smoothest result and is recommended. 

It is recommended to build up the Gaussian Process Regression model in u-space. Empirically found, this provides the best result.

\subsubsection{Update Gaussian Process Regression}

The uncertainty is used to select the next realization, which will be added to the trained model runs. The realization which produced the largest uncertainty will be used. There are a few options about the uncertainty. The preferred option is to use the "Wrong qualification", which is the probability that a wrong qualification (failing or not failing) is predicted.

