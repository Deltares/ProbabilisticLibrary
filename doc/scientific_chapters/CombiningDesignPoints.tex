\chapter{Combining failure probabilities}
\label{chp:CombiningFailureProbabilities}

The probabilities of failure of the components are combined to derive the probability of failure of the whole system. This is being referred to as system analysis. System analysis generally deals with parallel systems, series systems or combinations of both. A parallel system refers to a system in which failure only occurs if all components fail. A series system refers to a system where failure occurs if at least one of the components fails. This concept is schematically depicted in \Fref{fig:SystemAnalysis}.

\begin{figure}[H]\centering
\includegraphics*[width=5.96in, height=1.31in, keepaspectratio=false]{\pathProbLibPictures systemanalysis.png}
\caption{Schematic view of a parallel system, series system and a combination of both. Components $1$, $2$ and $3$ can be viewed as bridges. The systems fails if a passenger can not walk from left to right over the bridges.}\label{fig:SystemAnalysis}
\end{figure}

The general formulations of failure probabilities for parallel and series systems of $k$ components are as follows:
\begin{align}
&\text{Series:}  P_{f} =P\left[Z_{1} <0\cup ...\cup Z_{k} <0\right]=P\left[\bigcup _{i=1}^{k}Z_{i} <0 \right]=1-P\left[\bigcap _{i=1}^{k}Z_{i} \ge 0 \right] \label{eq:2-63} \\
&\text{Parallel:} P_{f} =P\left[Z_{1} <0\cap ...\cap Z_{k} <0\right]=P\left[\bigcap _{i=1}^{k}Z_{i} <0 \right]=1-P\left[\bigcup _{i=1}^{k}Z_{i} \ge 0 \right] \label{eq2-64} 
\end{align}

If the events $[Z_i<0]$, $i=1...k$ are mutually independent, this can be simplified to:
\begin{align}
&\text{Series:}  P_{f} =1-\prod _{i=1}^{k}\left\{1-P\left[Z_{i} <0\right]\right\}  \label{eq:2-65} \\
&\text{Parallel:} P_{f} =\prod _{i=1}^{k}P\left[Z_{i} <0\right]  \label{eq:2-66} 
\end{align}

The failure probabilities, $P[Z_i<0]$, for the individual components are determined by the probabilistic computation techniques as described in \autoref{chp:ReliabilityAnalysis}. System analysis for mutually independent components is therefore a relatively straightforward procedure. If the components are mutually correlated, the complexity of the system analysis increases. The correlations need to be taken into account as it increases the probability of failure of parallel systems and decreases the probability of failure of series systems. 

In the \probLib, failure probabilities of components are combined using the equivalent planes method, which is described in the following section.

\section{Equivalent planes}
\label{sec:EquivalentPlanes}

When combining afterwards, first design points are calculated for each failure definition individually and then they are combined.
This combination uses an equivalent plane $M$ for each design point, by which we make an approximation.
The equivalent plane is defined as follows:

\begin{align}
\label{eq:EquivalentPlane}
M_{\text{i}}\left(u\right) = \beta_{\text{i}} + \sum_{\text{j}} \alpha_{i,j} \cdot u_{\text{j}}
\end{align}

and 

\begin{align}
\label{eq:EquivalentPlaneCombine}
M_{\text{combined}}\left(u\right) = \left\{\begin{array}{ll}{\text{series}} & \min\limits_{\text{i}} M_{\text{i}}\left(u\right)\\
\text{parallel} & \max\limits_{\text{i}} M_{\text{i}}\left(u\right)
\end{array}	
\right.
\end{align}

where:

\begin{longtable*}{p{20mm}p{\textwidth-24pt-20mm}}  
	$M_{\text{i}}$ & is the result of the equivalent plane of design point $i$; \\
	$M_{\text{combined}}$ & is the result of the combination of a number of design points $i$; \\
	$\beta_{\text{i}}$ & is the reliability index of design point $i$; \\
	$\alpha_{i,j}$ & is the alpha factor of variable $j$ in design point $i$;
\end{longtable*}

\subsection{Directional sampling}
\label{sec:CombiningDesignPointsDS}

Directional sampling (see \autoref{sec:DirectionalSampling}) is applied on the combined model $M_{\text{combined}}$
(see \autoref{eq:EquivalentPlaneCombine}). The following settings are applied: 

\begin{longtable*}{p{40mm}p{\textwidth-40mm}}  
	Design point method & Center of gravity; \\
	Directions & 1000 - 10000; \\
	Convergence factor & 0.1;
\end{longtable*}

\subsection{Importance sampling}
\label{sec:CombiningDesignPointsIS}

Depending on the combination method (series or parallel) the following algorithms are applied:

\paragraph{Importance sampling - Series}

Importance sampling (see \autoref{sec:ImportanceSampling}) is applied on the combined model $M_{\text{combined}}$
(see \autoref{eq:EquivalentPlaneCombine}) in the following iterative way:
The combined probability of the first $N+1$ design points is the probability of the first $N+1$ design points added with the contribution of design point $N+1$,
where it does not have an overlap with one of the previous design points (see \autoref{eq:ImportanceSamplingSeries}).

\begin{align}
\label{eq:ImportanceSamplingSeries}
P\left(X_{1 \ldots N+1}\right) = P\left(X_{1 \ldots N}\right) + P\left(X_{N+1}\right) \cap P\left(\overline{X_{1 \ldots N}}\right)
\end{align}

where 

\begin{longtable*}{p{20mm}p{\textwidth-24pt-20mm}}  
	$P\left(X_{1 \ldots N+1}\right)$ & is the combined probability of the first $N+1$ design points; \\
	$P\left(\overline{X_{1 \ldots N}}\right)$ & is the complement of the combined probability of the first $N$ design points; \\
	$P\left(X_{N+1}\right)$ & is the probability of design point $N+1$;
\end{longtable*}

\autoref{eq:ImportanceSamplingSeries} can be rewritten as:

\begin{align}
\label{eq:ImportanceSamplingSeriesCond}
P\left(X_{1 \ldots N+1}\right) = P\left(X_{1 \ldots N}\right) + P\left(X_{N+1}\right) \cdot \left(1 - P\left(X_{1 \ldots N} | X_{N+1}\right)\right) 
\end{align}


The last term ($P\left(X_{1 \ldots N} | X_{N+1}\right)$) is calculated by importance sampling (see \autoref{sec:ImportanceSampling}).
The start point of the importance sampling is set to the design point of $X_{N+1}$.
The calculation is stopped when the remainder of the design points represent a small probability of failure.

\paragraph{Importance sampling - Parallel}

The parallel combination uses Crude Monte Carlo (see \autoref{sec:CrudeMonteCarlo}).
The sampling space is limited to the area which can lead to failure,
which could be very small.
Only values for $u_{\text{i}}$ are allowed for which the following condition is true:

\begin{align}
\forall k, u_{\text{min}} < u_{\text{j}} < u_{\text{min}} < u_{\text{j}} \exists u_{\text{i}} | M_{\text{k}}\left(u\right) = 0 \land u_{\text{min}} < u_{\text{i}} < u_{\text{max}}
\end{align}

where
\begin{longtable*}{p{20mm}p{\textwidth-24pt-20mm}}  
	$i$ & is the index of the $i^{\text{th}}$ variable; \\
	$j$ & is any other variable index ($j \neq i$); \\
	$k$ & is the index of the design point; \\
	$u_{\text{min}}$ & is the computationally minimum possible value of $u$ (-8); \\
	$u_{\text{max}}$ & is the computationally maximum possible value of $u$ (8); \\
	$M_{\text{k}}\left(u\right)$ & is the equivalent plane of the $i^{\text{th}}$ design point (see \autoref{eq:EquivalentPlane}).
\end{longtable*}

\subsection{Hohenbichler}\label{sec:Hohenbichler}

The kernel of the Hohenbichler algorithm is combination of two models $M_{\text{combined}}$ (see \autoref{eq:EquivalentPlaneCombine}).
This combination is calculated by FORM (see \autoref{sec:FORM}).

The result of this combination is a design point.
This design point is represented as an equivalent plane again when it should be combined with a third design point.
This will be repeated until all design points have been combined. 

First the two most contributing design points are combined,
then the remaining design with the highest probability if failure is added, and so on.
In this way the error made by representing intermediate design points is as low as possible.
