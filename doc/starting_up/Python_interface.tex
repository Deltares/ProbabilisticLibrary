\chapter{Python interface}
\label{ug:PythonInterface}

\section{Run python scripts}

The Probablistic Toolkit provides a python model, by which modifications can be made to a project file and runs can be made. This file is toolkit\_model.py is installed in the subdirectory Python of the installation directory of the Probabilistic Toolkit.

Python scripts can be run in two ways: Python is in control or the Probablistic Toolkit is in control

\subsection{Python in control}

In this way Python is the process which is started initially. The Python process makes connection to the Probabilistic Toolkit. Such a process consists of the following steps:

\begin{enumerate}
	\item Establish connection with the Probabilistic Toolkit;
	\item Load an input file (*.tkx), internally the contents of the file become available as a project
	\item Make changes to the project
	\item Run the calculation
	\item Retrieve the results of the calculation
	\item Save the results in a file (*.tkx)
	\item Disconnect from the Probablistic Toolkit
\end{enumerate}

This is coded as follows: 

\begin{lstlisting}[language=Python, basicstyle=\ttfamily, frame=single]
import sys
sys.path.append('<installdir>\Python')

from toolkit_model import *

# make connection to the Probabilistic Toolkit
toolkit = ToolKit()

# load an input file 
project = toolkit.load('example.tkx')

# make changes to the project, for example modify the settings
project.settings.method = 'FORM'
project.settings.relaxation_factor = 0.25

# validate and run if ok
messages = project.validate()
if len(messages) == 0:
	project.run()

# do something with the results, 
# for example print the reliabilty index
reliability_index = project.design_point.reliability_index
print('reliability index = ' + str(reliability_index))

# save the results
toolkit.save('example.tkx')

# disconnected when the toolkit object is garbage collected
\end{lstlisting}

\subsection{Toolkit in control}
\label{sec:PythonToolkitControl}

When the Probablistic Toolkit is started, it can run a python process before and after the calculation, see \autoref{sec:PrePostProcessing}. The connection with the toolkit is provided automatically and the project available in the toolkit is loaded already. The path to  toolkit\_model.py is set allready.

A preprocessor has in general the following structure:

\begin{enumerate}
	\item Connect to the project
	\item Make changes to the project
\end{enumerate}

In code a preprocessor will look as follows:
\begin{lstlisting}[language=Python, basicstyle=\ttfamily, frame=single]
from toolkit_model import *

# connect to the project
project = Project()

# do something with project, for example
# change one of the distributions to a fragility curve
h = project.model.get_variable('h')
h.clear()

h.set_fragility_prob_failure(1.5, 4.3)
h.set_fragility_prob_failure(2.8, 2,5)
\end{lstlisting}

After execution of this script, the Probablistic Toolkit will perform the calculation. Then a postprocessor can be invoked. A postprocessor will have the following structure:

\begin{enumerate}
	\item Connect to the project
	\item Retrieve the results of the calculation
\end{enumerate}

In code a postprocessor will look as follows:
\begin{lstlisting}[language=Python, basicstyle=\ttfamily, frame=single]
from toolkit_model import *

# connect to the project
project = Project()

# do something with the results, 
# for example print the reliabilty index
reliability_index = project.design_point.reliability_index
print('reliability index = ' + str(reliability_index))
\end{lstlisting}

or, when an output uncertainty calculation is made:

\begin{lstlisting}[language=Python, basicstyle=\ttfamily, frame=single]
import os
import sys
from toolkit_model import *

# argument for export file must be given
export_file = os.path.abspath(sys.argv[1])

project = Project()

lines = []

for var in project.model.response_variables:
	var = project.get_uncertainty_variable(var)
	lines.append("{0};{1};{2};{3}\n".format(
	  var.name,
	  var.get_quantile(0.1),
	  var.get_quantile(0.5),
	  var.get_quantile(0.9)))

with open(export_file, 'w') as f:
	f.writelines(lines)
\end{lstlisting}



\section{Reference}

The following methods are available:

\subsection{Class ToolKit}

The ToolKit class is only needed when Python is in control of the whole process. The Toolkit class should never be used in a preprocessor or postprocessor. The purpose of the class is to establish an connection with the Probabilistic Toolkit, loading a project and saving a project.

\begin{longtable*}{p{20mm}p{\textwidth-24pt-20mm}}  
	Constructor & \\
	Functionality & Establishes a connection with the Probabilistic Toolkit; \\  
	Arguments & None; \\  
\end{longtable*}

\begin{longtable*}{p{20mm}p{\textwidth-24pt-20mm}}  
	Finalizer & \\
	Functionality & Disconnects from the Probabilistic Toolkit; \\  
\end{longtable*}

\begin{longtable*}{p{20mm}p{\textwidth-24pt-20mm}}  
	Method &  \textbf{load}\\
	Functionality & Loads a project. All subsequent operations are applied on this project. Only one project can be loaded at a time; \\  
	Arguments & full path to *.tkx file; \\  
	Returns & Project; \\  
\end{longtable*}

\begin{longtable*}{p{20mm}p{\textwidth-24pt-20mm}}  
	Method &  \textbf{save}\\
	Functionality & Saves the current project; \\  
	Arguments & full path to *.tkx file; \\  
	Returns & Nothing; \\  
\end{longtable*}

\begin{longtable*}{p{20mm}p{\textwidth-24pt-20mm}}  
	Method &  \textbf{exit}\\
	Functionality & Optional method to stop the background server, which takes care for handling all commands. This call should not be necessary, since it is invoked automatically in the Finalizer. \\  
	Arguments & None; \\  
	Returns & Nothing; \\  
\end{longtable*}

\subsection{Class Project}

The Project class contains all input data and results of a project. It contains the methods to start the calculation.

\begin{longtable*}{p{20mm}p{\textwidth-24pt-20mm}}  
	Constructor & \\
	Functionality & Creates a project referencing the current project in the Probabilistic Toolkit. Only call the constructor if python is called in the preprocessor or postprocessor, otherwise use the load method in the class ToolKit; \\  
	Arguments & None; \\  
\end{longtable*}

\begin{longtable*}{p{20mm}p{\textwidth-24pt-20mm}}  
	Method &  \textbf{validate}\\
	Functionality & Validates the project; \\  
	Arguments & Nothing; \\  
	Returns & List of strings containing error validation messages; \\  
\end{longtable*}

\begin{longtable*}{p{20mm}p{\textwidth-24pt-20mm}}  
	Method &  \textbf{run}\\
	Functionality & Runs the calculation and waits until the calculation has finished. Do not call this method in a preprocessor or postprocessor; \\  
	Arguments & Nothing; \\  
	Returns & Indicator 'ok' or 'failed'; \\  
\end{longtable*}

\begin{longtable*}{p{20mm}p{\textwidth-24pt-20mm}}  
	Property &  \textbf{model}\\
	Functionality & Gets the model instance; \\  
	Type & Model; \\  
\end{longtable*}

\begin{longtable*}{p{20mm}p{\textwidth-24pt-20mm}}  
	Property &  \textbf{settings}\\
	Functionality & Gets an object containing all settings; \\  
	Type & Settings; \\  
\end{longtable*}

\begin{longtable*}{p{20mm}p{\textwidth-24pt-20mm}}  
	Property &  \textbf{identifier}\\
	Functionality & Gets or sets the identifier used in \autoref{sec:PrePostProcessing}; \\  
	Type & Model; \\  
\end{longtable*}

\begin{longtable*}{p{20mm}p{\textwidth-24pt-20mm}}  
	Property &  \textbf{uncertainty\_variable}\\
	Functionality & Gets the first uncertainty variable; \\  
	Type & UncertaintyStochast; \\  
\end{longtable*}

\begin{longtable*}{p{20mm}p{\textwidth-24pt-20mm}}  
	Method &  \textbf{get\_uncertainty\_variable}\\
	Functionality & Gets the uncertainty variable with a given name; \\  
	Arguments & string or ResponseStochast; \\  
	Returns & UncertaintyStochast; \\  
\end{longtable*}

\begin{longtable*}{p{20mm}p{\textwidth-24pt-20mm}}  
	Property &  \textbf{uncertainty\_variables}\\
	Functionality & Gets all resulting uncertainty variables; \\  
	Type & List of UncertaintyStochast; \\  
\end{longtable*}

\begin{longtable*}{p{20mm}p{\textwidth-24pt-20mm}}  
	Property &  \textbf{design\_point}\\
	Functionality & Gets the result of a reliability calculation; \\  
	Type & DesignPoint; \\  
\end{longtable*}

\begin{longtable*}{p{20mm}p{\textwidth-24pt-20mm}}  
	Property &  \textbf{design\_points}\\
	Functionality & Gets all resulting design points; \\  
	Type & List of DesignPoints; \\  
\end{longtable*}

\begin{longtable*}{p{20mm}p{\textwidth-24pt-20mm}}  
	Property &  \textbf{realizations}\\
	Functionality & Gets all realizations; \\  
	Type & List of Realizations; \\  
\end{longtable*}

\subsection{Class Model}

The Model class contains all input data. Do not create a Model, but use the Project.model property.

\begin{longtable*}{p{20mm}p{\textwidth-24pt-20mm}}  
	Property & \textbf{input\_file}\\
	Functionality & Gets or sets the input file name of a file based model; \\  
	Type & string (full path); \\  
\end{longtable*}

\begin{longtable*}{p{20mm}p{\textwidth-24pt-20mm}}  
	Property &  \textbf{submodels}\\
	Functionality & Gets a list of all sub models in case of a composite model; \\  
	Type & string (full path); \\  
\end{longtable*}

\begin{longtable*}{p{20mm}p{\textwidth-24pt-20mm}}  
	Method &  \textbf{get\_submodel}\\
	Functionality & Gets the submodel with a specified name; \\  
	Arguments & name of variable; \\  
	Returns & SubModel or None if not found; \\  
\end{longtable*}

\begin{longtable*}{p{20mm}p{\textwidth-24pt-20mm}}  
	Property &  \textbf{variables}\\
	Functionality & Gets a list of all variables; \\  
	Type & array of strings; \\  
\end{longtable*}

\begin{longtable*}{p{20mm}p{\textwidth-24pt-20mm}}  
	Method &  \textbf{get\_variable}\\
	Functionality & Gets the variable with a specified full name of the variable (name including model name if the model is a composite model); \\  
	Arguments & name of variable; \\  
	Returns & Stochast or None if not found; \\  
\end{longtable*}

\begin{longtable*}{p{20mm}p{\textwidth-24pt-20mm}}  
	Property &  \textbf{response\_variables}\\
	Functionality & Gets a list of all response variables; \\  
	Type & string (full path); \\  
\end{longtable*}

\begin{longtable*}{p{20mm}p{\textwidth-24pt-20mm}}  
	Method &  \textbf{get\_response\_variable}\\
	Functionality & Gets the response variable with a specified name; \\  
	Arguments & name of variable; \\  
	Returns & ResponseStochast or None if not found; \\  
\end{longtable*}

\begin{longtable*}{p{20mm}p{\textwidth-24pt-20mm}}  
	Method &  \textbf{run}\\
	Functionality & Runs the model with the input file specified for this model. This might be necessary to generate the output values in the output file, which is necessary for the Probabilistic Toolkit to perform an project.run(); \\  
	Arguments & None; \\  
	Returns & Nothing; \\  
\end{longtable*}

\subsection{Class SubModel}

The SubModel class contains input data of a sub model. Do not create this class.

\begin{longtable*}{p{20mm}p{\textwidth-24pt-20mm}}  
	Property &  \textbf{input\_file}\\
	Functionality & Gets or sets the input file name of a file based model; \\  
	Type & string (full path); \\  
\end{longtable*}

\begin{longtable*}{p{20mm}p{\textwidth-24pt-20mm}}  
	Method &  \textbf{run}\\
	Functionality & Runs the model with the specified for this submodel. This might be necessary to generate the output values in the output file, which is necessary for the Probabilistic Toolkit to perform an project.run(); \\  
	Arguments & None; \\  
	Returns & Nothing; \\  
\end{longtable*}

\subsection{Class Stochast}

The Stochast class contains all properties of a variable.

\begin{longtable*}{p{20mm}p{\textwidth-24pt-20mm}}  
	Property &  \textbf{name}\\
	Functionality & Gets the name of the variable; \\  
	Type & string; \\  
\end{longtable*}

\begin{longtable*}{p{20mm}p{\textwidth-24pt-20mm}}  
	Property &  \textbf{fullname}\\
	Functionality & If the model is a composite model, gets the model name and variable name, else gets the variable name; \\  
	Type & string; \\  
\end{longtable*}

\begin{longtable*}{p{20mm}p{\textwidth-24pt-20mm}}  
	Property &  \textbf{distribution}\\
	Functionality & Gets or sets the distribution type, one of:  Deterministic, Normal, LogNormal, StudentT, Uniform, Triangular, Trapezoidal, Exponential, Gamma, Beta, Frechet, Weibull, Gumbel, GeneralizedExtremeValue, Rayleigh, Pareto, GeneralisedPareto, Table (= histogram), Discrete, Poisson, FragilityCurve). The distribution type is not case sensitive; \\  
	Type & string; \\  
\end{longtable*}

\begin{longtable*}{p{20mm}p{\textwidth-24pt-20mm}}  
	Properties &  \textbf{mean, deviation, variation, minimum, maximum, shift, shift\_b, rate, shape, shape\_b, location, scale, observations, design\_fraction, design\_factor}\\
	Functionality & Gets or sets the property of a variable. Deviation refers to standard deviation and variation refers to variation coefficient; \\  
	Type & float; \\  
\end{longtable*}

\begin{longtable*}{p{20mm}p{\textwidth-24pt-20mm}}  
	Method & \textbf{get\_quantile}\\
	Functionality & Gets the value belonging to a given quantile; \\  
	Arguments & Quantile; \\  
	Returns & Value at quantile; \\  
\end{longtable*}

\begin{longtable*}{p{20mm}p{\textwidth-24pt-20mm}}  
	Method & \textbf{get\_design\_value}\\
	Functionality & Gets the design value; \\  
	Arguments & None; \\  
	Returns & Value based on design\_fraction and design\_factor; \\  
\end{longtable*}

\begin{longtable*}{p{20mm}p{\textwidth-24pt-20mm}}  
	Method &  \textbf{clear}\\
	Functionality & Resets the distribution type to deterministic and removes all fragility values, discrete values and histogram bins; \\  
	Arguments & None; \\  
	Returns & Nothing; \\  
\end{longtable*}

\begin{longtable*}{p{20mm}p{\textwidth-24pt-20mm}}  
	Methods & \textbf{set\_fragility\_reliability\_index, set\_fragility\_prob\_failure, set\_fragility\_prob\_non\_failure} \\
	Functionality & Adds or updates a fragility value of a variable which has distribution type fragility curve; \\  
	Arguments & Variable name; \\  
	& Value of the conditional variable; \\
	& Reliability index, probabilty of failure or non failure at the value of the conditional variable; \\
	Returns & Nothing; \\  
\end{longtable*}

\begin{longtable*}{p{20mm}p{\textwidth-24pt-20mm}}  
	Method & \textbf{set\_fragility\_reliability\_index} \\
	Functionality & Adds or updates a fragility value of a variable which has distribution type fragility curve; \\  
	Arguments & Variable name; \\  
	& Value of the conditional variable; \\
	& Reliability index at the value of the conditional variable; \\
	Returns & Nothing; \\  
\end{longtable*}

\begin{longtable*}{p{20mm}p{\textwidth-24pt-20mm}}  
	Method & \textbf{set\_discrete\_value} \\
	Functionality & Adds or updates a discrete value of a variable which has distribution type discrete; \\  
	Arguments & Variable name; \\  
	& Value of the conditional variable; \\
	& Occurrences at the value of the conditional variable; \\
	Returns & Nothing; \\  
\end{longtable*}

\begin{longtable*}{p{20mm}p{\textwidth-24pt-20mm}}  
	Method & \textbf{set\_histogram\_value} \\
	Functionality & Adds a bin of a variable which has distribution type histogram; \\  
	Arguments & Variable name; \\  
	& Lower boundary of the bin; \\
	& Upper boundary of the bin; \\
	& Number of occurences in the bin; \\
	Returns & Nothing; \\  
\end{longtable*}

\begin{longtable*}{p{20mm}p{\textwidth-24pt-20mm}}  
	Methods & \textbf{get\_correlation, set\_correlation} \\
	Functionality & Gets or sest the correlation between two variables; \\  
	Arguments & Other variable name; \\
	& set: new correlation value\\  
	Returns & get: Correlation, set: Nothing; \\  
\end{longtable*}

\subsection{Class ResponseStochast}

The Stochast class contains all properties of a response variable.

\begin{longtable*}{p{20mm}p{\textwidth-24pt-20mm}}  
	Property &  \textbf{name}\\
	Functionality & Gets the name of the response variable; \\  
	Type & string; \\  
\end{longtable*}

\begin{longtable*}{p{20mm}p{\textwidth-24pt-20mm}}  
	Property &  \textbf{fullname}\\
	Functionality & Gets the full name (name including model if the model is a composite model) of the response variable; \\  
	Type & string; \\  
\end{longtable*}

\subsection{Class Settings}

The Settings class contains computational settings.

\begin{longtable*}{p{20mm}p{\textwidth-24pt-20mm}}  
	Property & \textbf{method} \\
	Functionality & Gets or sets the calculation method; \\  
	Type & Calculation method (one of: NumericalIntegration, NumericalBisection, MonteCarlo, LatinHyperCube, DirectionalSampling, ImportanceSampling, SubsetSimulation, Cobyla, FORM, FOSM, FragilityCurveIntegration, 	Experimental). The calculation method is not case sensitive; \\  
\end{longtable*}

\begin{longtable*}{p{20mm}p{\textwidth-24pt-20mm}}  
	Property & \textbf{start\_method} \\
	Functionality & Sets the calculation start method; \\  
	Type & Calculation start method (one of: None, RaySearch, SensitivitySearch, SphereSearch). The calculation start method is not case sensitive; \\  
\end{longtable*}

\begin{longtable*}{p{20mm}p{\textwidth-24pt-20mm}}  
	Properties & \textbf{minimum\_samples, maximum\_samples, minimum\_iterations, maximum\_iterations, relaxation\_factor, relaxation\_loops, variation\_coefficient\_failure, intervals, variance\_factor, start\_value, variance\_loops,	min\_variance\_loops, fraction\_failed} \\
	Functionality & Gets or sets a settings value; \\  
	Type & float\\  
\end{longtable*}

\begin{longtable*}{p{20mm}p{\textwidth-24pt-20mm}}  
	Method & \textbf{get\_variable\_settings} \\
	Functionality & Gets an object containing calculation settings of a variable; \\  
	Arguments & Variable name; \\  
	Returns & StochastSettings; \\  
\end{longtable*}

\subsection{Class StochastSettings}

The StochastSettings class contains computational settings of a variable

\begin{longtable*}{p{20mm}p{\textwidth-24pt-20mm}}  
	Property & \textbf{name} \\
	Functionality & Gets the variable name; \\  
	Type & string; \\  
\end{longtable*}

\begin{longtable*}{p{20mm}p{\textwidth-24pt-20mm}}  
	Property & \textbf{start\_value} \\
	Functionality & Gets or sets the start value; \\  
	Type & float; \\  
\end{longtable*}

\subsection{Class UncertaintyStochast}

The DesignPoint class contains the result of a reliability calculation

\begin{longtable*}{p{20mm}p{\textwidth-24pt-20mm}}  
	Property & \textbf{name} \\
	Functionality & Gets the name of the reponse variable; \\  
	Type & string; \\  
\end{longtable*}

\begin{longtable*}{p{20mm}p{\textwidth-24pt-20mm}}  
	Method & \textbf{get\_quantile} \\
	Functionality & Gets the quantile of the uncertainty variable; \\  
	Arguments & Quantile; \\  
	Returns & Value at quantile; \\  
\end{longtable*}


\subsection{Class DesignPoint}

The DesignPoint class contains the result of a reliability calculation

\begin{longtable*}{p{20mm}p{\textwidth-24pt-20mm}}  
	Property & \textbf{identifier} \\
	Functionality & Gets the numeric identifier of the design point, for example the varying parameters in a table scenario; \\  
	Type & array of floats; \\  
\end{longtable*}

\begin{longtable*}{p{20mm}p{\textwidth-24pt-20mm}}  
	Properties & \textbf{reliability\_index, probability\_failure, convergence} \\
	Functionality & Gets the result values of the reliability calculation; \\  
	Type & float; \\  
\end{longtable*}

\begin{longtable*}{p{20mm}p{\textwidth-24pt-20mm}}  
	Method & \textbf{get\_alpha} \\
	Functionality & Gets the contribution of a variable to the design point; \\  
	Arguments & Variable full name or ResponseStochast; \\  
	Returns & Alpha; \\  
\end{longtable*}

\begin{longtable*}{p{20mm}p{\textwidth-24pt-20mm}}  
	Property & \textbf{realizations} \\
	Functionality & Gets the realizations needed to calculate the design point; \\  
	Type & List of Realizations; \\  
\end{longtable*}

\subsection{Class Alpha}

\begin{longtable*}{p{20mm}p{\textwidth-24pt-20mm}}  
	Properties & \textbf{alpha\_value, physical\_value} \\
	Functionality & Gets the result values of the contribution to the design point; \\  
	Type & float; \\  
\end{longtable*}

\subsection{Class Realization}

\begin{longtable*}{p{20mm}p{\textwidth-24pt-20mm}}  
	Properties & \textbf{z, weight, beta} \\
	Functionality & Gets the limit state, weight and beta of the realization; \\  
	Type & float; \\  
\end{longtable*}

\begin{longtable*}{p{20mm}p{\textwidth-24pt-20mm}}  
	Method & \textbf{get\_value} \\
	Functionality & Gets the value of a variable (input or response) in the realization; \\  
	Arguments & Full name of the variable; \\  
	Returns & Value of the stochastic variable in the realization; \\  
\end{longtable*}

