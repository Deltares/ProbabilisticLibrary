\pagestyle{empty}
\cleardoublepage
\pagestyle{empty}

\reference{11209269-008-GEO-0009}
\keywords{probabilistic library, WBI2023, BOI, functional design, (non-)functional requirements}
\summary{This document describes the \textit{Functional design} of the Probabilistic Library \textbf{version \probLibVersion{}}, part of BOI (in Dutch: \textit{Beoordelings- en Ontwerp Instrumentarium}). The BOI program derives guidelines, methods and tools for the statutory safety assessment and design of flood defences in the Netherlands. The Probabilistic Library provides a set of routines that enable system reliability analyses. The general goal of system reliability analyses is the derivation of the probability of failure of a system consisting of multiple components. The functional design describes functional and non-functional requirements imposed on the Probabilistic Library.}

%\versioni{1.0}
%\datei{Dec 2019}
%\authori{Karolina \newline Wojciechowska}
%\revieweri{Ferdinand \newline Diermanse \newline Tom The \newline Hans van Putten}
%\approvali{Jan Aart \newline van Twillert}
%
%\versionii{2.0}
%\dateii{Feb 2021}
%\authorii{Karolina \newline Wojciechowska}
%\reviewerii{Ferdinand \newline Diermanse \newline Rob Prevel}
%\approvalii{Jan Aart \newline van Twillert}
%
%\versioniii{3.0}
%\dateiii{May 2022}
%\authoriii{Karolina \newline Wojciechowska}
%\revieweriii{Ferdinand \newline Diermanse \newline Rob Prevel}
%\approvaliii{Jan Aart \newline van Twillert}

\versioni{4.0}
\datei{Juni 2023}
\authori{Karolina \newline Wojciechowska}
\revieweri{Arjen Markus \newline Marcel Pastoors}
\approvali{Jan Aart \newline van Twillert}

\status{Final}

\signatureGB

\pagestyle{empty}
\cleardoublepage
\pagestyle{empty}

\reference{11209269-008-GEO-0009}
\keywords{probabilistische bibliotheek, WBI2023, BOI, functioneel ontwerp, (niet-)functionele eisen}
\summary{Dit document beschrijft het \textit{Functioneel ontwerp} van de Probabilistische Bibliotheek \textbf{versie \probLibVersion{}}, onderdeel van BOI (Beoordelings- en Ontwerp Instrumentarium). Het BOI-programma bepaalt richtlijnen, methoden en instrumenten voor de wettelijke veiligheidsbeoordeling en ontwerpen van waterkeringen in Nederland. De probabilistische bibliotheek biedt een set routines, die analyses van de systeembetrouwbaarheid mogelijk maken. Het algemene doel van systeembetrouwbaarheidsanalyses is het afleiden van de faalkans van een systeem dat uit meerdere componenten bestaat. Het functionele ontwerp beschrijft functionele en niet-functionele eisen, die aan de Probabilistische Bibliotheek worden opgelegd.}

%\versioni{1.0}
%\datei{Dec 2019}
%\authori{Karolina \newline Wojciechowska}
%\revieweri{Ferdinand \newline Diermanse \newline Tom The \newline Hans van Putten}
%\approvali{Jan Aart \newline van Twillert}
%
%\versionii{2.0}
%\dateii{Feb 2021}
%\authorii{Karolina \newline Wojciechowska}
%\reviewerii{Ferdinand \newline Diermanse \newline Rob Prevel}
%\approvalii{Jan Aart \newline van Twillert}
%
%\versioniii{3.0}
%\dateiii{May 2022}
%\authoriii{Karolina \newline Wojciechowska}
%\revieweriii{Ferdinand \newline Diermanse \newline Rob Prevel}
%\approvaliii{Jan Aart \newline van Twillert}

\versioni{4.0}
\datei{June 2023}
\authori{Karolina \newline Wojciechowska}
\revieweri{Arjen Markus \newline Marcel Pastoors}
\approvali{Jan Aart \newline van Twillert}

\status{Definitief}

\signatureNL